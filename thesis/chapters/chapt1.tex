\chapter{Introductie}

"Aanbevelingssystemen zijn software tools en technieken die aanbevelingen voor items voorzien die nuttig zijn voor de gebruiker. Deze aanbevelingen voorzien door het aanbevelingssysteem hebben de bedoeling de gebruikers te ondersteunen in het maken van keuzes, zoals welk item te kopen, welke muziek te beluisteren en welke nieuwsberichten te lezen." \cite{recsys_handbook}
Aanbevelingssystemen zijn alomtegenwoordig in ons dagelijks leven. Entertainmentproviders zoals Spotify en Netflix gebruiken al jaren aanbevelingssystemen om ons kennis te laten maken met nieuwe items. Google gebruikt het onder meer in Maps, om lokale bedrijven en horeca aan te rijken aan de gebruiker.


Er zijn verschillende motieven om aanbevelingssystemen te gebruiken.
Door betere content aan te bieden zorgt het ervoor dat de gebruikers langer een platform gebruiken. Langer op TikTok scrollen betekent dat de gebruiker meer advertenties zal bekijken. \cite{tiktokalgorithm}
Ook laat het de gebruiker kennis maken met dikkestaart-items (Eng: long-tail items). Dit zijn de niche items die dus minder populair zijn, maar daarom niet minder kwalitatief. Doordat ze minder populair zijn, zijn ze vaak minder aanwezig in een fysieke winkel. Een online winkel heeft deze limitatie niet, en kan met behulp van aanbevelingssystemen steeds het optimale product weergeven voor de specifieke gebruiker. \cite{cursus_hs2} \mijnfiguur[H]{width=12cm}{fig/chapt1/long_tail.png}{De dikke staart}{fig:dikke_staart}

Webwinkels kunnen aanbevelingssystemen ook gebruiken om commercieel interessantere items een voorkeur te laten genieten. Zo kunnen items met hogere winstmarges vaker aanbevolen worden.


\section{Probleemstelling}
Als een gebruiker op restaurant wilt gaan eten, beperkt die zich vaak tot de keuzes die hij al kent. Dit fenomeen kunnen we linken aan het verankeringseffect \cite{anchoring_effect}, waarbij een persoon te veel waarde hecht aan de enkele restaurants die hij al heeft uitgeprobeerd. Er zijn echter veel restaurants die zeer goed bij de gebruiker zouden passen, maar waar hij geen weet van heeft.

\section{Bestaande oplossingen}
Diensten zoals Tripadvisor \cite{tripadvisor_algorithm} of Google Maps helpen een gebruiker om deze keuze te maken. Deze diensten maken gebruik van aanbevelingssystemen om restaurants aan te bieden aan gebruikers op basis van diens locatie, voorkeuren en zoekterm. Echter zijn de gebruikte systemen niet feilloos. We bespreken de onderliggende algoritmen in \autoref{sec:huidige_technieken_aanbevelingssystemen}.

\section{Doelstelling}
% TODO: doelstelling: wat willen we bereiken?

% TODO: dit hoofdstuk verder uitbreiden met informatie uit hs2 van de slides van recsys, de problemen moeten pas in volgende hoofdstuk aangekaart worden
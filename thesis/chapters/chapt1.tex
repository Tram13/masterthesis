\chapter{Introductie}

"Aanbevelingssystemen zijn software tools en technieken die aanbevelingen voor items voorzien die nuttig zijn voor de gebruiker. Deze aanbevelingen voorzien door het aanbevelingssysteem hebben de bedoeling de gebruikers te ondersteunen in het maken van keuzes, zoals welk item te kopen, welke muziek te beluisteren en welke nieuwsberichten te lezen." \cite{recsys_handbook}
Aanbevelingssystemen zijn alomtegenwoordig in ons dagelijks leven. Entertainmentproviders zoals Spotify en Netflix gebruiken al jaren aanbevelingssystemen om ons kennis te laten maken met nieuwe items. Google gebruikt het onder meer in Maps, om lokale bedrijven en horeca aan te rijken aan de gebruiker.

De keuze over welke producten relevant zijn voor een specifieke gebruiker gebeurt aan de hand van gegevens over de producten, en eventueel gegevens over de gebruiker. Stel het voorbeeld van een gsm-winkel: bij producten houden we steeds de schermgrootte, prijsklasse en opslagcapaciteit bij. Voor de gebruiker slaan we een aankoopgeschiedenis op. Zo kunnen we voorkeuren leren uit de aankoopgeschiedenis en deze als filter toepassen op het aanbod van gsm's.

Er zijn verschillende motieven om aanbevelingssystemen te gebruiken.
Door betere content aan te bieden zorgt het ervoor dat de gebruikers langer een platform gebruiken. Langer op TikTok scrollen betekent dat de gebruiker meer advertenties zal bekijken. \cite{tiktokalgorithm}
Ook laat het de gebruiker kennis maken met dikkestaart-items (Eng: long-tail items). Dit zijn de niche items die dus minder populair zijn, maar daarom niet minder kwalitatief. Doordat ze minder populair zijn, zijn ze wegens plaatsgebrek minder aanwezig in een fysieke winkel. Een online winkel heeft deze beperking niet. Door de enorme toename aan keuze voor de online klant is filteren veel belangrijker geworden. Hierdoor zijn aanbevelingssystemen noodzakelijk bij grote webwinkels \cite{rise_of_recsys_in_ecommerce}. Zo kan steeds het optimale product weergegeven worden aan iedere individuele gebruiker. \cite{cursus_hs2}

Webwinkels kunnen aanbevelingssystemen ook gebruiken om commercieel interessantere items een voorkeur te laten genieten. Zo kunnen items met hogere winstmarges vaker aanbevolen worden.

\mijnfiguur[H]{width=12cm}{fig/chapt1/long_tail.png}{De dikke staart}{fig:chapt1_dikke_staart}

\section{Probleemstelling}
Als een gebruiker op restaurant wilt gaan eten, beperkt die zich vaak tot de keuzes die hij al kent. Dit fenomeen kunnen we linken aan het verankeringseffect \cite{anchoring_effect}, waarbij een persoon te veel waarde hecht aan de enkele restaurants die hij al heeft uitgeprobeerd. Er zijn echter veel restaurants die zeer goed bij de gebruiker zouden passen, maar waar hij geen weet van heeft.

\section{Bestaande oplossingen}
Diensten zoals Tripadvisor \cite{tripadvisor_algorithm}, Yelp of Google Maps helpen een gebruiker om deze keuze te maken. Deze diensten maken gebruik van aanbevelingssystemen om restaurants aan te bieden aan gebruikers op basis van diens locatie, voorkeuren en zoekterm. Echter zijn de systemen die gebruikt worden door de grote spelers niet feilloos \cite{spotify_recsys_bad}. We bespreken de onderliggende algoritmen in \autoref{sec:huidige_technieken_aanbevelingssystemen}.

\section{Doelstelling}
Ons onderzoek tracht met dezelfde beschikbare data als andere algoritmen een betere ervaring voor de gebruiker te creëren door beter in te kunnen schatten wat de gebruiker belangrijk vindt bij een restaurantbezoek. Deze betere ervaring komt neer op preciezer de score te voorspellen die de gebruiker aan een restaurant zou geven. Als we accuraat scores kunnen voorspellen, kunnen we de gebruiker de restaurants met de hoogste verwachte scores aanrijken.
% TODO: online test doen of niet? Anders komt er hier nog bij dat we ook het gevoel van de gebruikers gaan bevragen in een online test.

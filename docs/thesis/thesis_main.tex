\documentclass[a4paper, usecolor, twoside, onehalfspacing, 11pt]{bwthesis}
% The usecolor option sets the titles in blue, as requested by
% the Ghent University housestyle. Remove this to get a black
% and white version of things.
\usepackage[dutch]{babel} %Use dutch headings and titles
\usepackage{hyperref}

\addto\extrasdutch{%
  \def\sectionautorefname{Sectie}%
  \def\subsectionautorefname{Sectie}%
  \def\subsubsectionautorefname{Sectie}%
}


%-------------------------------------------------------------------------------
% FILL IN YOUR DETAILS
%
% Keep in mind that UGent doesn't use copromotors any longer. However,
% if it is required to add, please uncomment the line starting with
% \copromotor below and fill in the correct details.
%
\title{Design van een machine learning-algoritme voor aanbevelingen van restaurants op basis van gelabelde en tekstuele data}
%\subtitle{}
\author{Arnoud De Jonge en Arno Vermote}

%\wordcount{6.431} % Fill in the number of words if needed
\studentnr{01808870 - 01806792} % Fill in your student number

\promotor{prof. dr. ir. Toon De Pessemier, prof. dr. ir. Luc Martens}
%\copromotor{Dr. FirstName LastName}
\tutor{prof. dr. ir. Toon De Pessemier}

\degree{master} % Mogelijkheid: bachelor of master
\richting{Informatica}

\academicyear{2022 - 2023}

%-------------------------------------------------------------------------------
% The preamble. Note that the following packages are already loaded by
% the class bwthesis: geometry, amsmath, amsfonts, amssymb, graphicx,
% xcolor, ulem, setspace

% This package provides the lstlisting environment
\usepackage{listings}
\input{listingfile} % contains settings for the package listings
% this package provides an environment for algorithms (cfr. Pseudocode)
\usepackage[ruled]{algorithm2e} 
% This package provides extra possibilities for tables
\usepackage{booktabs}
% this package is used to produce both author-date and standard numerical citations for BibTeX bibliographies
\usepackage[defernumbers=true]{biblatex}
\bibliography{Thesis_bib}
\usepackage{csquotes}
\usepackage{float}
\usepackage{alltt}
\usepackage{amsmath}
\usepackage{graphicx}
\usepackage{subcaption}
\usepackage{multicol}
\usepackage{sidecap}
\setlength{\columnsep}{5mm} % column separation


\usepackage{array}
\newenvironment{conditions}
  {\par\vspace{\abovedisplayskip}\noindent\begin{tabular}{>{$}l<{$} @{${}={}$} l}}
  {\end{tabular}\par\vspace{\belowdisplayskip}}
  

\usepackage{fancyvrb}
\newcommand\verbatimbold[1]{\textbf{#1}}
% This packages adds the Appendix name in the toc. See below
\usepackage[titletoc]{appendix}

% ---- ADDITIONAL SETTINGS
\graphicspath{{fig/}} % path to the figure directory

% In the preamble, you can also define your own extra commands:
%---------------------------------------------------------------------------
% Short version commando to introduce figures. 
%---------------------------------------------------------------------------
%\mijnfiguur[H]{width=5cm}{bestandsnaam}{Het bijschrift bij deze figuur}{label}
\newcommand{\mijnfiguur}[5][ht]{            % Het eerste argument is standaard `ht'.
    \begin{figure}[#1]                      % Beginnen van de figure omgeving
        \begin{center}                      % Beginnen van de center omgeving
            \includegraphics[#2]{#3}        % Het eigenlijk invoegen van de figuur (2: opties, 3: bestandsnaam)
        \end{center}
        \caption{#4}          % Het bijschrift (argument 4) en het label (argument 5)
				\label{#5}
    \end{figure}
    }

%-------------------------------------------------------------------------------
% The actual document
%


\begin{document}


% Typisch copyright voor een thesis.
% Te plaatsen juist na het titelblad.
% De namen worden automatisch ingevuld, maar 

\par\vspace*{\fill}

De auteurs en promotor geven de toelating deze scriptie voor consultatie beschikbaar te stellen en delen ervan te kopi\"eren voor persoonlijk gebruik. Elk ander gebruik valt onder de beperkingen van het auteursrecht, in het bijzonder met betrekking tot de verplichting uitdrukkelijk de bron te vermelden bij het aanhalen van resultaten uit deze scriptie.

The authors and promoter give the permission to use this thesis for consultation and to copy parts of it for personal use. Every other use is subject to the copyright laws, more specifically the source must be extensively specified when using results from this thesis.

\vspace{1cm}

Gent, \today % TODO Vul de juiste datum in!!!

\vspace{1cm}

\begin{minipage}[t][4cm][t]{0.5\textwidth}
\raggedright
The promotors,

\vspace{2.5cm}

prof. dr. ir. Toon De Pessemier

prof. dr. ir. Luc Martens
\end{minipage}
\begin{minipage}[t][4cm][t]{0.48\textwidth}
\raggedright
The authors,

\vspace{2.5cm}

Arnoud De Jonge

Arno Vermote
\end{minipage}

\thispagestyle{empty}


\clearpage{\pagestyle{empty}\cleardoublepage}

%------------------------------------------------------------------------
\frontmatter
\pagestyle{frontmatter} %sets headers and footers correctly

% ------------ thanks -----------
\chapter{Dankwoord}

Merci aan allen!



% ------------ table of contents ---------
{
	\singlespacing % to keep the TOC within boundaris
  % Verhinder witruimte tussen secties
  \setlength{\parskip}{0ex plus 0.3ex minus 0.3ex}
	\tableofcontents
}

\addcontentsline{toc}{chapter}{Inhoudsopgave} %add TOC to the TOC


% ------------ summary ----------
\chapter[Nederlandstalige samenvatting]{Samenvatting}
% TODO: Volgens UGent Guidelines is dit 1 pagina. Echter is onze thesis met 2 personen, dus 2 pagina's? Ik zou gaan voor 1 zeeeeeer goed gevulde pagina.
% TODO: GEEN GRAFIEKEN OF TABELLEN OFZO, TEKSTUEEL!

% TODO: Inleiding van probleem

% TODO: Probleem- en doelstelling

% TODO: Bestaande technieken, en dat ze nog steeds fouten maken, 2 tot 4 zinnen max!

% TODO: NLP, en hoe we uiteindelijk de tekst omzetten naar features (enkel wat we uiteindelijk gebruiken), hoe we objectief vergelijken met andere implementaties (testmethode) (ik zou zeggen hier zo weinig mogelijk zeggen over """gefaalde""" experimenten)

% TODO: MLP: welke architectuur lijkt het beste te werken? + testmethode

% TODO: Wat zijn de uiteindelijke resultaten?

% TODO: vergelijking met bestaande methoden + conclusie





% TODO: VERGEET NIET STEEDS BRONVERMELDINGEN!





{\titlespacing*{\chapter}
{0pt}{-5pt}{-5pt}\chapter[Summary]{}}
\rule{\linewidth}{0.5mm}
\hspace*{0pt}
\section*{Design of a machine learning algorithm for restaurant \\recommendations based on labeled and textual data}
\rule{\linewidth}{0.5mm}

\textbf{Students:}
\hfill
\textbf{Supervisors:} \\
Arnoud De Jonge
\hfill
prof. dr. ir. Toon De Pessemier \\
Arno Vermote
\hfill
prof. dr. ir. Luc Martens \\
\\
\hspace*{0pt}\hfill Master Computer Science


\begin{otherlanguage}{english}
\begin{multicols}{2}
[
\section*{\centering Abstract}
Recommender systems provide assistance to end users in making decisions. Existing state-of-the-art algorithms use either labeled or textual data. We combined both data sources to increase the accuracy of the predicted scores. On the Yelp Dataset we measured an RMSE of 1.1107, which is better than the Wide \& Deep Learning RMSE of 1.4025 and the DeepCoNN RMSE of 1.1642. Predicting reviews of 1/5 stars or 5/5 stars remains a problem, due to the limited training samples of these classes. 
]

\section*{1. Introduction}
Recommender systems are used by many service providers and webshops to provide a smart filtering of items. Often, these recommendations are personalised to match the preferences of the users. This is done by collecting data for every user and aggregating this into user profiles. Similarly, all data of each item is aggregated into an item profile. Then, these profiles are compared to identify if the item matches the preferences of the user.

The state of the art consists of DeepCoNN \cite{deepconn_eng_summary} for recommendations based on purely textual data. This will often be written reviews from customers. The state of the art for recommendations based on labeled data is Wide \& Deep Learning \cite{wide_deep_learning_paper_eng_summary}.\newline
In this paper we explore the potential performance gains when utilizing both data sources to predict the score a specified user gives to a specified item. The domain of the predicitons will be scores for restaurants. The data is sourced from Yelp. \cite{Yelp_Dataset_eng_summary}

\section*{2. Methodologies}
We will first transform the textual data into user- and item profiles using Natural Language Processing (NLP) techniques. Then, these profiles will be combined with profiles created using the labeled data. The result will be fed into a neural network, which will make a prediction for the expected score of that user for that restaurant.

\subsection*{Labeled Data Profiles}
These profiles are constructed using the labels available in the Yelp dataset. This dataset assigns each restaurant a collection of \q{categories} and \q{attributes}. These labels are then one-hot encoded in a vector, which represents the restaurant profile. The user profiles are then constructed using a sum for each label of the rated restaurants, multiplied by their normalized score minus 0.5 (neutral score after normalization). This way we model the user profile with the same labels as the restaurant profiles, but with the user's preference included.

\subsection*{Profiles From Text}
We create a second pair of profiles, but now using the written reviews as data source. For this, we will use BERTopic. \cite{bertopic_paper_eng_summary} This is mainly an offline algorithm, but online and other variations exist. BERTopic's structure consists of multiple parts: each part can be swapped out individually for another similar method. As a result of this, the algorithm is highly customizable. Therefore constructing an adaptation is almost trivial. The first part is the embeddingsmodel. This will convert the sentences into numeric vectors. For the embeddingsmodel we will use transformer models, a relatively new discovery in NLP. To be precise, we are using Sentence-BERT \cite{sentence_bert_eng_summary}. \newline
After generating a numeric vector for each sentence, a dimensionality reduction algorithm reduces the dimension of that vector. This will be done to tackle the challenges that comes with high-dimensional data \cite{curse_of_dim_eng_summary}. Any dimensionality reduction algorithm such as UMAP or PCA can be used. For the online variant we chose an adaptation to PCA, namely Incremental PCA \cite{ipca_eng_summary}.\newline
After the dimension is reduced, the algorithm will produce a clustering. Once again multiple clustering algorithms can be used, including HDBSCAN and K-Means. In our experiments we utilize MiniBatch K-Means \cite{kmeans_minibatch_eng_summary} for the online variant. \newline
The final step is to construct a representation for every cluster. This is done by creating a bag-of-words representation for the combined text of all documents in one cluster. Afterwards c-TF-IDF \cite{bertopic_c_tf_idf_eng_summary} is applied. This is a class based variant of TF-IDF. The final representation of one cluster or topic consists of the most frequent words defined by c-TF-IDF. It is possible to finetune these representations even more, for example by using KeyBERT \cite{keybert_eng_summary}.

\subsection*{Creating Predictions}
A neural network is a class of supervised machine learning models. It is inspired by the way a human brain works, with neurons connected to each other in layers. Often, neural networks have the most potential to achieve the lowest loss of all machine learning methods after extensive training. \cite{cursus_ML_supervised_eng_summary} Neural networks have the downside that it is difficult to interpret or explain the decision making process.\newline
The combined input data as calculated in the previous steps is provided to a neural network, with all profiles for one specific restaurant and one specific user at a time. The neural network then provides a score, wich corresponds to the predicted rating which the user would give to that restaurant.

\section*{3. Experimental Setup}
\subsection*{BERTopic}
To use textual reviews in a neural network, we must convert these into a numeric inputvector. We focus on the added value of the topic modelling algorithm BERTopic. The model takes multiple textual documents as input to generate a clustering. Each cluster is then represented by a set of words. 

In our case we use the sentences from the reviews as input for the BERTopic model. Given the large amount of reviews, we expect to reach the limitations of system memory using the standard BERTopic algorithm. Consequently, we investigate the performance of a better scaling online BERTopic variant. Using various BERTopic models, we can manufacture the profiles of users and restaurants. This is done for every user and restaurant based on the reviews they gave or received. We evaluate the different building blocks to implement a BERTopic model: we will compare the profiles that are based on the clustering of a model, to profiles created by solely using the representations. Another interesting hyperparamter is the amount of clusters, which is equivalent to the length of the final profiles. Finally we will measure the impact of using subjective predefined topics to assist the model training. Note that in this last experiment, the model is free to adapt or dismiss inaccurate topics.

Additionally we experiment with sentiment analysis \cite{sentiment_transformer_paper_eng_summary}, which works independently from the BERTopic algorithm. Sentiment analysis is used to extract positive or negative user experiences from a written review without any additional information. Reviews can contain positive and negative feedback at the same time. Combining this with BERTopic to determine the subject can result in an improved feature vector.

To evaluate the results we will use a neural network. This will be a simple neural network with 5 hidden layers. To compare different profiles, all other parameters remain fixed. The combination of profiles that provides the minimal loss will be considered better. Since a neural network is a detour to evaluate a BERTopic model, we also use an alternative method: to directly evaluate the clustering we make use of clustering metrics such as the silhouette index or Davies-Bouldin index. Finally we analyse if the same patterns occur between the clustering metrics and the loss.


\subsection*{Neural Network}
After the best performing combination of input profiles is found as explained in the previous sections, we try to adapt the neural network to further enhance its predictive capabilities. We experiment with multiple parameters to create an optimal neural network architecture, based on the general performance, measured in MSE. After finding the ideal model using the MSE criterium, edge case performance and rigidity of the network is tested. We treat each parameter as fully independent to all other parameters. This assumption is not entirely sound, but conducting a full grid search for every combination of parameters would be too computationally expensive. While this assumption might not provide us with the overall best possible network, it will likely suffice for all practical purposes.

We explore the amount of layers in the neural network: we implement different networks from one up to eight hidden layers. Each hidden layer halves the amount of neurons. The final output layer has one neuron. The models with five or more hidden layers have a small adaptation, where they will first increase the amount of neurons up to 150\% of the amount of neurons in the input vector, to then scale down gradually to one neuron in the output layer.\newline
We also experiment with the more dynamic ADAGRAD optimizer, compared to the basic SGD optimizer. After finding the best performing network configuration, we measure the effect of the size of the dataset on the performance and the effect of the cold start problem on the ability to provide accurate predictions.\newline
To validate the performance comes from combining the textual and labeled data sources, we also trained a Random Forest model with this data. We then compare our final results to other algorithms: traditional implementations as state-of-the-art techniques.


\section*{4. Results and Discussion}
\subsection*{NLP profiles}
We can conclude that an online BERTopic model heavily outperforms the offline standard. We did not discover any significant differences when comparing profiles based on clustering against the profiles based on the representation. Also, the addition of predefined topics yields similar results. When increasing the amount of clusters from 50 to 400 we see a marginal increase in performance, yet not any remarkable improvements. The addition of sentiment does significantly increase the performance. The top performance is gained when only applying sentiment to restaurant profiles. \newline
Finally, we observe a correlation in the loss and results of the clusteringmetrics. This means that it is possible to determine if a BERTopic model has an acceptable clustering without training a neural network. Note that these metrics can be influenced when using higher dimensions without having the same impact on the loss. In addition, metrics are not applicable to NLP analyses that are independent from BERTopic, e.g.: sentiment analysis.


\subsection*{Rating Predicions}
As discovered in the previous section, the input parameter will consist of two NLP profiles with each a dimension of 400, combined with two profiles extracted from the labeled dataset. In total, the inputvector has a dimension of $1000$. The optimal network architecture to handle this input exists of six hidden layers, but networks with seven or eight hidden layers also perform well.\newline
SGD was not able to sufficiently train the network. The use of the ADAGRAD optimizer is necessary. A learning rate of 0.0002 seemed optimal. The network could be sufficiently trained and tested with only 50\% of the data set. With even smaller data sets, the model would start to sacrifice generality.\newline
The effect of the cold start problem is limited. For users with less than five reviews we were able to predict ratings with an accuracy of 22\%, compared to an accuracy of 36\% for users with more than 25 reviews. However, we noticed the network really struggles to predict the less represented classes. We measured an accuracy of only 1.7\% for 1-star reviews (\autoref{fig:eng_summ_acc_1star}). This is because the network tends to fall back to predicting 4 stars when it is uncertain, since this value provides a middle ground. We believe further research into this problem is required.

Finally, we were able to beat both DeepCoNN and Wide \& Deep Learning in terms of RMSE (\autoref{fig:eng_summ_RMSE}). The RMSE represents the average difference between the provided score by the user and the predicted score (out of 5 stars). This confirms our hypthesis that the combination of multiple data sources can lead to better predictions.

\begin{figure}[H]
    \begin{center}
        \includegraphics[width=8cm]{fig/eng_summary/accuracy_1star.png}
    \end{center}
    \caption{Prediction Accuracy of 1-star Reviews}
    \label{fig:eng_summ_acc_1star}
\end{figure}

\begin{figure}[H]
    \begin{center}
        \includegraphics[width=8cm]{fig/eng_summary/comparison_implementations.png}
    \end{center}
    \caption{Comparing of RMSE for different implementations \cite{deepconn_eng_summary, wide_deep_learning_paper_eng_summary}}
    \label{fig:eng_summ_RMSE}
\end{figure}

\printbibliography[heading=subbibliography, keyword={eng_summary}]
\end{multicols}
\end{otherlanguage}


% The following can be commented out to remove the list of figures
% and the list of tables, as specified by the guidelines of BW
% \listoffigures
% \listoftables

%-----------------------------------------------------------------------
\mainmatter
\pagestyle{mainmatter} % sets headers and footers correctly

% Here you can add more chapters in case it is needed
\chapter{Introductie}

"Aanbevelingssystemen zijn softwaretools en -technieken die aanbevelingen voor items voorzien die nuttig zijn voor de gebruiker. Deze aanbevelingen voorzien door het aanbevelingssysteem hebben de bedoeling de gebruikers te ondersteunen in het maken van keuzes, zoals welk item te kopen, welke muziek te beluisteren en welke nieuwsberichten te lezen." \cite{recsys_handbook}
Aanbevelingssystemen zijn alomtegenwoordig in ons dagelijks leven. Entertainmentproviders zoals Spotify en Netflix gebruiken al jaren aanbevelingssystemen om ons kennis te laten maken met nieuwe muziek en films. Google gebruikt het onder meer in Maps, om lokale bedrijven en horeca aan te rijken aan de gebruiker.

De keuze over welke producten relevant zijn voor een specifieke gebruiker gebeurt aan de hand van gegevens over de producten, en eventueel gegevens over de gebruiker. Stel het voorbeeld van een gsm-winkel: bij producten beschikt de winkel over gegevens die de schermgrootte, prijsklasse en opslagcapaciteit beschrijven. Voor de gebruiker houdt die een aankoopgeschiedenis bij. Zo kunnen we voorkeuren leren uit de aankoopgeschiedenis en deze als filter toepassen op het aanbod van gsm's.

Er zijn verschillende motieven om aanbevelingssystemen te gebruiken. Als social media-platformen betere content aanbieden aan gebruikers zorgt dit voor een hogere screentime. Langer op TikTok scrollen betekent dat de gebruiker meer advertenties bekijkt, en dus meer inkomsten genereert. \cite{tiktokalgorithm}
Ook laat het de gebruiker kennis maken met dikkestaart-items (Eng: long-tail items). Dit zijn de niche items die dus minder populair zijn, maar daarom niet minder kwalitatief. Doordat ze minder populair zijn, zijn ze wegens plaatsgebrek minder aanwezig in een fysieke winkel. Een online winkel heeft deze beperking niet. Er is echter zodanig veel keuze, dat de gebruiker overweldigd wordt. Deze enorme hoeveelheid aan producten creëert net een barrière voor de gebruiker. \cite{paradox_choice} Daarom is het slim filteren van items veel belangrijker geworden. Hierdoor zijn aanbevelingssystemen noodzakelijk bij grote webwinkels \cite{rise_of_recsys_in_ecommerce}. Zo kan steeds het optimale product weergegeven worden aan iedere individuele gebruiker. \cite{cursus_hs2}

Webwinkels kunnen aanbevelingssystemen ook gebruiken om commercieel interessantere items een voorkeur te laten genieten. Zo kunnen items met hogere winstmarges vaker aanbevolen worden.

\mijnfiguur[H]{width=12cm}{fig/chapt1/long_tail.png}{De dikke staart}{fig:chapt1_dikke_staart}

\section{Probleemstelling}
Als een gebruiker op restaurant wilt gaan eten, beperkt die zich vaak tot de keuzes die hij al kent. Dit fenomeen kunnen we linken aan het verankeringseffect \cite{anchoring_effect}, waarbij een persoon te veel waarde hecht aan de enkele restaurants die hij al heeft uitgeprobeerd. Er zijn echter veel restaurants die zeer goed bij de gebruiker zouden passen, maar waar hij geen weet van heeft.

Diensten zoals Tripadvisor \cite{tripadvisor_algorithm}, Yelp of Google Maps helpen een gebruiker om deze keuze te maken. Deze diensten maken gebruik van aanbevelingssystemen om restaurants aan te bieden aan gebruikers op basis van diens locatie, voorkeuren en zoekterm. Echter zijn de systemen die gebruikt worden door de grote spelers niet feilloos \cite{recsys_bad, recsys_youtube_bad}. 

Er is nog weinig onderzoek naar aanbevelingssystemen die een combinatie van labels en vrije, tekstuele data zoals geschreven reviews gebruiken. Meer specifiek, er ontbreekt onderzoek naar aanbevelingssystemen die transformermodellen gebruiken om extra features toe te voegen aan een machine-learning model. Afzonderlijk onderzoek naar technieken die transformermodellen gebruiken om features te maken voor aanbevelingen bestaat al, maar die features worden niet gebruikt voor machine-learning systemen. \cite{masterthesis_nlp_italie} Aan de andere kant bestaan aanbevelingssystemen die volledig op machine learning gebaseerd zijn, maar geen transformermodellen gebruiken. \cite{deepconn} In deze thesis onderzoeken we of deze combinatie leidt tot een betere modellering van de voorkeuren van de gebruiker en of hieruit betere voorspellingen volgen. De onderliggende algoritmen van de belangrijkste en meest gebruikte technieken bespreken we in \autoref{sec:chapt2_huidige_technieken_aanbevelingssystemen}. 


\section{Doelstelling}
Ons onderzoek tracht met dezelfde beschikbare data als andere algoritmen een betere ervaring voor de gebruiker te creëren door beter in te kunnen schatten wat de gebruiker belangrijk vindt bij een restaurantbezoek. Deze betere ervaring komt neer op preciezer de score te voorspellen die de gebruiker aan een restaurant zou geven. Hiervoor gebruiken we een hybride aanbevelingssysteem, waarbij de geschreven reviews als extra databron worden gebruikt om meer informatie over de gebruiker te capteren. Met transformermodellen wordt deze extra data omgezet tot nieuwe numerieke features, die dan samengevoegd worden bij de gelabelde data (feature augmentation). Zo creëren we voorstellingen voor gebruikers en restaurants. Deze worden dan als input gebruikt voor een neuraal netwerk, waarbij de output een verwachte score voorstelt die de gebruiker aan dat restaurant zou geven.
% TODO: type netwerk dat gebruikt is? Is dit een MLP of gebeurt hier nog iets speciaal?
% TODO: eventueel verwijzingen maken naar de relevante hoofdstukken hier? Zoals 'omgezet naar numerieke features' --> hs 4, deel Arnoud
Als we accuraat scores kunnen voorspellen, kunnen we de gebruiker de restaurants met de hoogste verwachte scores aanrijken, waardoor de gebruikerservaring en vertrouwen in het aanbevelingssysteem groeit.
% TODO: online test doen of niet? Anders komt er hier nog bij dat we ook het gevoel van de gebruikers gaan bevragen in een online test.
\chapter{Huidige technieken}
\label{chap:huidige_technieken}
% TODO: overzicht van wat al bestaat
\section{Aanbevelingssystemen}
\label{sec:huidige_technieken_aanbevelingssystemen}
In essentie probeert een aanbevelingssysteem te voorspellen welke producten een gebruiker nuttig zal vinden. Dit gebeurt in de meeste toepassingen \cite{overzicht_technieken} door te voorspellen welke score een gebruiker aan ieder item zou toekennen, en dan de best scorende producten terug te geven. 

We kunnen dit formeel noteren als
\begin{equation}
U \times I \rightarrow \hat{R}
\label{def:chap2_aanbevelingssysteem_formeel}    
\end{equation}
waarbij $U$ een vector is die de gebruikers voorstelt, $I$ een vector is die de items voorstelt en $\hat{R}$ de verwachte scores zijn. \cite{cursus_hs2} $\hat{R}$ is dan een matrix, waarbij iedere kolom overeenkomt met een item en iedere rij overeenkomt met een gebruiker.

\begin{table}[H]
\centering
\begin{tabular}{c|ccc}
        & $Item_0$ & $Item_1$ & $Item_2$ \\ \hline
$User_0$ & 0.5     & 0.6     & 0.7     \\
$User_1$ & 0.8     & 0.8     & 0.9     \\
$User_2$ & 0.3     & 0.9     & 0.8    
\end{tabular}
\caption{Voorbeeld voor $\hat{R}$ met fictieve data}
\end{table}

Hieruit volgt dat een top $N$ beste producten voor een gebruiker neerkomt op de volgende berekening:
\begin{lstlisting}
    scores = []
    for item in items:
        scores.append(score(user, item))
    scores.sort_desc()
    scores[0:N]
\end{lstlisting}

Het design van een aanbevelingssysteem kan gezien worden als een optimalisatieprobleem waarbij we $|(R - \hat{R})|$ minimaliseren, met $R$ de effectieve scores zijn die de gebruikers zouden toekennen aan de items.

Er zijn dus 3 factoren die invloed hebben op de accuraatheid $|(R - \hat{R})|$ van een aanbevelingssysteem: $U$, $I$ en de operator $\times$, die $U$ en $I$ verwerkt tot een score. $U$ en $I$ zijn gebaseerd op de oorspronkelijke data, en worden met feature engineering-technieken omgezet tot numerieke features. De $\times$-operator kan op veel verschillende manieren deze features combineren tot een voorspelling $\hat{R}$. Bij het ontwerp van een aanbevelingssysteem is het dus belangrijk om deze 3 parameters te bestuderen.

% TODO: praten over non-personalised recommenders, maar dat we dat negeren door te slechte performance obviously
% TODO: Bespreking van verschillende technieken: basis tem SotA. Vrijwel enkel de werking, niet de voor-nadelen
% TODO: intro hier
\subsection{Niet-gepersonaliseerde systemen}
\label{sec:chapt2_non_persionalised}
Niet-gepersonaliseerde aanbevelingssystemen gebruiken geen gegevens over de gebruiker om aanbevelingen te maken. Met andere woorden, $U$ is de eenheidsvector. Er wordt enkel beroep gedaan op data van de producten, zoals het aantal verkochte exemplaren of het aantal positieve reviews. Verschillende metrieken kunnen met feature engineering gecombineerd worden om zo betere resultaten te bekomen.

\subsection{Gebruikersprofielen}
Het is voor een aanbevelingssysteem uitermate belangrijk om de voorkeuren van een gebruiker goed in te kunnen schatten. Bij veel methoden wordt er per gebruiker een 'gebruikersprofiel' opgesteld: dit profiel is een vector waarvan iedere dimensie een eigenschap van een product of gebruiker voorstelt. Het opstellen van een gebruikersprofiel gebeurt impliciet aan de hand van de aankoopgeschiedenis/reviews... van de gebruiker. Het is ook mogelijk om de gebruiker in een vragenlijst expliciet om zijn voorkeuren te vragen.

\begin{table}[H]
\centering
\begin{tabular}{c|ccc}
         & $Property_0$ & $Property_1$ & $Property_2$ \\ \hline
$User_0$ & 0.2          & 0            & 0.7          \\
$User_1$ & 0.1          & 0.8          & 0.6          \\
$User_2$ & 0.9          & 0.9          & 0.2         
\end{tabular}
\caption{Voorbeeld voor $U$ met fictieve data}
\label{tab:chap2_user_profiles}
\end{table}

Door \autoref{tab:chap2_user_profiles} is het duidelijk dat in de praktijk $U$ een matrix is in definitie \ref{def:chap2_aanbevelingssysteem_formeel}. Dit zal zo zijn voor iedere techniek die gebruikersprofielen gebruikt, ongeacht hoe die profielen worden opgesteld.

\subsection{Traditionele methoden}
\label{sec:chapt2_traditionele_methoden}
Er bestaan verschillende technieken om aanbevelingssystemen te implementeren. Traditionele algoritmen zoals Content-Based Filtering (CB) en Collaborative Filtering (CF) zijn wijd toepasbaar in verschillende contexten. "CB en CF werken door prioriteiten toe te kennen aan de beschikbare informatie en hierop te filteren." \cite{overzicht_technieken} Voor al deze technieken is er steeds een éénduidig gedefinieerde operator $\times$.


\subsubsection{Content-Based Filtering}
Deze techniek is gebaseerd op de metadata van de producten. Er wordt per gebruiker een profiel aangemaakt, dat de voorkeuren voor eigenschappen van producten weerspiegeld. Toegepast op een aanbevelingssysteem voor restaurants zijn deze eigenschappen bijvoorbeeld de prijsklasse, keuken en kindvriendelijkheid. Hoe meer metadata beschikbaar is, hoe preciezer de voorkeuren van de gebruiker gemodelleerd kunnen worden. Het gebruikersprofiel wordt dan vergeleken met alle beschikbare items, om zo de items die het dichtste aansluiten bij het gebruikersprofiel aan te bieden. Formeel geldt bij Content-Based Filtering voor gebruiker $i$:


\begin{equation}
    U_i = \sum_{n=1}^{N} I_n
    \label{eq:chap2_cb_user_profile}
\end{equation}
met $N$ het aantal producten en $I_n$ een vector die de aanwezigheid van iedere mogelijke eigenschap aanduidt. Dit gebruikersprofiel kan dan vergeleken worden met ieder product via de cosinusgelijkenis $S_C$:
\begin{equation}
    S_C(U_i, I_j) = \frac{U_i \cdot I_j}{\Vert U_i \Vert \Vert I_j \Vert}
    \label{eq:chap2_cb_cosine_similarity}
\end{equation}

Er bestaan heel veel alternatieve formules voor het opstellen van gebruikersprofielen. Er kan op verschillende plaatsen genormaliseerd worden en technieken zoals Term Frequenqy - Inverse Document Frequency (TF-IDF) kunnen toegepast worden op de eigenschappen. Scores kunnen herschaald worden om negatieve waarden toe te kennen aan eigenschappen of producten met negatieve scores kunnen lagere gewichten krijgen. De optimale combinatie van technieken hangt steeds af van het probleem.

\subsubsection{Collaborative Filtering}
\label{sec:chapt2_cf}
Bij Collaborative Filtering maken we geen gebruik van metadata. Bij User-User Collaborative Filtering (UUCF) kijken we in de plaats naar het gedrag van andere gebruikers. Hierbij worden opnieuw gebruikersprofielen opgesteld, zoals in definitie \ref{eq:chap2_cb_user_profile}. Hierna worden deze met elkaar vergeleken met Pearsons correlatiecoëfficiënt $C_p$:

\begin{equation}
    C_p(U_i, U_j) = \frac{\sum_{k = 1}^{m}(r_{i, k} - \overline{r_i})(r_{j, k} - \overline{r_j})}{\sqrt{\sum_{k = 1}^{m}(r_{i, k} - \overline{r_i})^2} \sqrt{\sum_{k = 1}^{m}(r_{j, k} - \overline{r_j})^2}}
    \label{eq:chapt2_pearson_corr}
\end{equation}
\cite{UUCF_original_paper}, waarbij $r_{i, k}$ de score voorstelt die gebruiker $i$ gaf aan product $k$.
Pearsons correlatiecoëfficiënt is een veralgemening van de cosinusgelijkenis (\ref{eq:chap2_cb_cosine_similarity}). Er bestaan nog verschillende variaties \cite{UUCF_alternative_implementations} op deze formule die bijvoorbeeld gebruik maken van normalisatie en significance weighting \cite{CF_significance_weighting}. Dit laatste is een techniek waarbij twee gebruikers die weinig gemeenschappelijke items hebben een lagere score krijgen.

Na het berekenen van Pearsons correlatiecoëfficiënt kunnen nu aanbevelingen gegenereerd worden. De aanbevelingen voor gebruiker $i$ komen dan uit gebruiker(s) $j$, waarvoor geldt:
\begin{equation}
    C_p(U_i, U_j) = \max_{k \in U}(C_p(U_i, U_k))    
    \label{eq:chapt2_neighbour_calculation}
\end{equation}

We noemen $I_j$ dan een buur van $I_i$. UUCF veronderstelt dat gelijk gedrag in het verleden wijst op gelijk gedrag in de toekomst. Ook stelt UUCF dat dat de niet-overlappende interessedomeinen van twee buren toch interessant zijn voor elkaar. UUCF gaat er dus impliciet van uit dat de overlap van interesses volledig is (\autoref{fig:chapt2_user_profiles_overlap}).

\mijnfiguur[H]{width=12cm}{fig/chapt2/user_profiles_overlap.png}{Visualisatie overlap interesses van twee buren}{fig:chapt2_user_profiles_overlap}

Het aantal buren dat in rekening wordt gebracht kan variëren tussen implementaties. Het is mogelijk om een top $K$ buren te nemen en dan de verwachte score voor een product $p$ als volgt te berekenen:
\begin{equation}
    \hat{r}_{i, p} = \overline{r_i} + \frac{\sum_{u=1}^{K}(r_{u, p} - \overline{r_u}) \cdot C_p(U_i, U_u)}{\sum_{u=1}^{K} C_p(U_i, U_u)}
    \label{eq:chapt2_uucf_finding_predictions_from_neighbours}
\end{equation}
In definitie \ref{eq:chapt2_neighbour_calculation} is het aantal buren $K = 1$. Een groter aantal buren zorgt voor een stabieler maar minder specifiek resultaat door het toevoegen van ruis in het beslissingsproces. \cite{cursus_hs8}

Een andere variant van Collaborative Filtering is Item-Item Collaborative Filtering (IICF). Waar UUCF gelijkaardige gebruikers met elkaar verbindt, zal IICF gelijkaardige items zoeken. In tegenstelling tot CF (\autoref{sec:chapt2_cf}) gebruiken we hiervoor geen metadata \cite{IICF_original_paper}. We kijken in de plaats naar de andere items die ook gekozen werden door gebruikers die het oorspronkelijke item kozen. Als een item door veel andere gebruikers ook gekozen werd, noemen we dat item een buur van het oorspronkelijke item. Net zoals UUCF, gaat IICF er van uit dat de voorkeuren van een gebruiker stabiel blijven, zodanig dat de buren steeds relevant blijven \cite{cursus_hs9}. In de praktijk bestaan er 'seizoensgebonden' items, zoals een kerstbar, maar deze zijn eerder uitzonderlijk.

Om producten met elkaar te vergelijken, maken we opnieuw gebruik van Pearsons correlatiecoëfficiënt, analoog als in formule \ref{eq:chapt2_pearson_corr}. We vervangen dan de paren van gebruikers naar paren van items.
Intuïtief komt UUCF overeen met het zoeken naar vergelijkbare rijen en IICF met het zoeken naar vergelijkbare kolommen in \autoref{tab:chapt2_uucf_iicf_example}:

\begin{table}[H]
\centering
\begin{tabular}{c|ccc}
         & $Item_0$ & $Item_1$ & $Item_2$ \\ \hline
$User_0$ & 0.2      & 0        & 0.7      \\
$User_1$ & 0.1      & 0.8      & 0.6      \\
$User_2$ & 0.9      & 0.9      & 0.2     
\end{tabular}
\caption{Voorbeeld voor $R$ met fictieve data}
\label{tab:chapt2_uucf_iicf_example}
\end{table}

In de praktijk hebben niet alle gebruikers alle items een score gegeven, en zullen dus niet alle elementen van $R$ ingevuld zijn. Ook bij IICF bestaan er verschillende varianten op Pearsons correlatiecoëfficiënt om de gelijkheid tussen twee items te bepalen. Analoog aan definities \ref{eq:chapt2_neighbour_calculation} en \ref{eq:chapt2_uucf_finding_predictions_from_neighbours} kunnen het vereist aantal buren gevonden worden en de verwachte scores voor nieuwe producten berekend worden \cite{IICF_original_paper}.

Een groot voordeel van IICF aanbevelingssystemen is de schaalbaarheid bij grote itemsets. Als er veel verschillende items zijn, is het bij UUCF niet altijd mogelijk om een buur te vinden die dat specifieke item al een score heeft gegeven. In dat geval is het dus niet mogelijk om een score voor de huidige gebruiker te voorspellen. Als itemset $\gg$ userset, dan stelt dit probleem zich niet bij IICF. In de praktijk is dit een vaker voorkomend scenario \cite{recsys_handbook}.

\subsection{Methoden gebaseerd op machine learning}
De afgelopen jaren is er een explosie aan nieuwe technieken voor aanbevelingssystemen gebaseerd op machine learning (ML). Dit is ook zichtbaar in \autoref{fig:chapt2_research_trend_recsys_ml}.

\mijnfiguur[H]{width=12cm}{fig/chapt2/recommender-system-machine-learning_edit.png}{Stijgend aantal publicaties over Recommender Systems en ML \cite{recsys_ml_popularity}}{fig:chapt2_research_trend_recsys_ml}

 Deze nieuwe technieken zijn vaak in staat om significant accuratere nieuwe scores te voorspellen. Dit gaat echter ten koste van 'explainability', of de mogelijkheid om te verklaren waarom het aanbevelingssysteem een specifieke score voorspelt \cite{overzicht_technieken}. Machine learning-technieken zijn vaak ook geoptimaliseerd voor een specifiek probleem met een specifieke dataset, en vereisen dat tot tientallen hyperparameters worden gefinetuned bij een implementatie in een nieuwe context. Het zijn ook de enigste technieken die ongestructureerde data zoals tekst of afbeeldingen kunnen verwerken en omzetten naar kennis. Zo bestaat er bijvoorbeeld een aanbevelingssysteem dat zich toespitst op het aanbevelen van social media posts, gebaseerd op de tekst van de post en de inhoud van de bijhorende foto \cite{recsys_afbeeldingen_social_network}.

 Uit tientallen papers kozen we twee state-of-the-art algoritmen om te bespreken. We kozen deze op basis van volgende criteria:
\begin{itemize}
     \item Recente datum van publicatie
     \item Goede performantie over verschillende datasets
     \item Gebruik van machine learning
     \item Implementatie in code beschikbaar
     \item Link met eigen onderzoek (op basis van tekst of labels)
\end{itemize}

De eerste paper is 'Joint Deep Modeling of Users and Items Using Reviews for Recommendation' (2017) waarin de DeepCoNN-architectuur wordt voorgesteld \cite{deepconn}. DeepCoNN gebruikt enkel geschreven reviews om aanbevelingen op te stellen. De architectuur bestaat uit twee parallelle neurale netwerken. Het ene netwerk verwerkt de reviews, gegroepeerd per gebruiker. Op die manier wordt een gebruikersprofiel gemaakt. Analoog verwerkt het tweede neurale netwerk alle reviews, gegroepeerd per item. Zo wordt dan een itemprofiel gemaakt. Slechts in de laatste laag van het DeepCoNN-netwerk worden de parallelle netwerken met elkaar verbonden via een fully-connected layer en wordt de loss berekend.

\mijnfiguur[H]{width=12cm}{fig/chapt2/deepconn_architectuur.png}{DeepCoNN Architectuur \cite{deepconn}}{fig:chapt2_deepconn_architecture}

De tweede paper die we bestuderen is 'Wide \& Deep Learning for Recommender Systems' (2016) \cite{wide_deep_learning_paper}. De input van het Wide \& Deep-netwerk zijn gewone labels. Het 'wide' gedeelte verwijst naar een simpel lineair neuraal netwerk. Dit netwerk kan eenvoudige, expliciete feature-interacties modelleren. Aan de andere kant bestaat het 'deep'-component: hiermee kunnen de complexere, non-lineaire interacties tussen features gemodelleerd worden. 

Op het einde worden de 'Wide' en 'Deep' netwerken gecombineerd in een fully-connected layer om een score te berekenen en een aanbeveling te maken. De architectuur staat visueel weergegeven in \autoref{fig:chapt2_deep_wide_architecture}

\mijnfiguur[H]{width=12cm}{fig/chapt2/wide_deep_architectuur.png}{Wide \& Deep Architectuur \cite{wide_deep_learning_paper}}{fig:chapt2_deep_wide_architecture}

\subsection{Hybride modellen}

Hybride modellen, of ensemble modellen, implementeren meerdere technieken in één model. Er bestaat zes hybridisatiemodellen:

\begin{table}[H]
\centering
\begin{tabular}{ll}
\multicolumn{1}{l|}{Hybridisatiemodel} & Beschrijving \\ \hline
\multicolumn{1}{l|}{} &  \\
\multicolumn{1}{l|}{Gewichten} & \begin{tabular}[c]{@{}l@{}}De scores van verschillende technieken combineren\\ met een gewicht voor één product\end{tabular} \\
\multicolumn{1}{l|}{} &  \\
\multicolumn{1}{l|}{Wisselen} & \begin{tabular}[c]{@{}l@{}}Om de beurt een andere techniek gebruiken per\\ product\end{tabular} \\
\multicolumn{1}{l|}{} &  \\
\multicolumn{1}{l|}{Features combineren} & \begin{tabular}[c]{@{}l@{}}De werkwijze van de ene techniek nabootsen\\ in de andere techniek\end{tabular} \\
\multicolumn{1}{l|}{} &  \\
\multicolumn{1}{l|}{Feature augmentatie} & \begin{tabular}[c]{@{}l@{}}De score van de ene techniek wordt gebruikt\\ toegevoegd aan de input van een andere techniek\end{tabular} \\
\multicolumn{1}{l|}{} &  \\
\multicolumn{1}{l|}{Cascade} & \begin{tabular}[c]{@{}l@{}}De ene techniek toepassen op een subset van de\\ items gegenereerd door de andere techniek\end{tabular} \\
\multicolumn{1}{l|}{} &  \\
\multicolumn{1}{l|}{Meta-level} & \begin{tabular}[c]{@{}l@{}}Het aangeleerde model van de ene techniek wordt\\ gebruikt als input bij de andere techniek\end{tabular} \\
 & 
\end{tabular}                                                 
\caption{Verschillende hybridisatiemodellen voor aanbevelingssystemen \cite{hybrid_recsys_models, cursus_hs11}}
\label{tab:chapt2_hybridisatiemodellen}
\end{table}

Het doel van een hybridemodel is zwaktes elimineren van alleenstaande modellen. Stel een voorbeeld van een hybride aanbevelingssysteem met gewichten 0.7 voor IICF en 0.3 voor CB. We weten dat in de meeste gevallen IICF beter zal presteren. Daarom krijgt het een hogere score. Echter, we hebben gemeten dat in sommige gevallen IICF een compleet foute aanbeveling maakt. Door het gebruik van een hybride aanbevelingssysteem kan deze fout opgevangen worden door de CB recommender die dan een zeer lage score zal toewijzen, waardoor de gewogen eindscore van dit slechte product toch laag zal zijn en niet aanbevolen zal worden.

Het is duidelijk dat het gebruik van hybridemodellen zowel de accuraatheid als consistentie van een aanbevelingssysteem kan verbeteren. Een hybridemodel correct toepassen vereist wel een zeer goed begrip in de alleenstaande technieken en een goed inzicht in de omstandigheden waarin deze technieken soms falen. In de praktijk zijn hybridemodellen op basis van gewichten de meest voorkomende implementatie. \cite{hybrid_recsys_literature_overview}

De Netflix Prize competitie daagde onderzoekers uit om het inhouse aanbevelingsalgoritme Cinematch te verslaan in RMSE. Het winnende team kon zo 1 miljoen USD binnenhalen. BigChaos, een hybride model dat bestond uit meer dan 100 verschillende algoritmen, won deze competitie met een  10\% lagere RSME dan Cinematch. \cite{netflix_hybrid}

\subsection{Uitdagingen}
De technieken in \autoref{sec:huidige_technieken_aanbevelingssystemen} staan beschreven in chronologische volgorde. Iedere techniek is steeds een evolutie op de vorige door de accuraatheid, snelheid, schaalbaarheid... te verbeteren. Echter kunnen we niet zeggen dat in alle gevallen de nieuwste methode de beste is. Aanbevelingssytemen zijn vaak zeer gevoelig aan de context waarin ze gebruikt worden, en het doel dat voor ogen is.


\subsubsection{Cold-Startprobleem}
\label{sec:chapt2_cold_start}
"Het Cold-Startprobleem beschrijft de problematiek van het maken van aanbevelingen wanneer de gebruiker of het item nieuw is." \cite{coldstart_cf} We kunnen dit probleem dus in twee subproblemen opdelen: nieuwe gebruikers en nieuwe items.

Bij nieuwe items zullen CB-technieken weinig problemen ondervinden, daar deze onmiddellijk aan de hand van de metadata kunnen gelinkt worden aan bestaande items. Bij UUCF is dit moeilijker: daar wordt een item pas aanbevolen indien het geconsumeerd wordt door buren van een gebruiker. Doordat het een nieuw item is, heeft geen enkele gebruiker het item geconsumeerd, en wordt het dus ook bij geen enkele buur aanbevolen. Er kan een analoog besluit gevormd worden voor IICF. Collaborative Filtering heeft traditioneel dus geen oplossing voor het Cold-Startprobleem \cite{recsys_diversity}.  Bij machine learning-modellen hangt de invloed van het Cold-Startprobleem vast aan de gebruikte features bij de input. Hoe meer features afhangen van het aantal reviews/scores over het item, hoe slechter het zal presteren. Doordat DeepCoNN meer informatie uit weinig geschreven reviews kan halen dan de Wide \& Deep Learning-architectuur, is het DeepCoNN hier minder gevoelig aan. \cite{deepconn}

Nieuwe gebruikers vormen een groter probleem: als we niets weten over de voorkeuren van een gebruiker, is het moeilijk om een persoonlijke aanbeveling te maken. Om dit op te lossen, kan een hybride aanbevelingssysteem ingezet worden. Dit hybride model bevat dan onder andere een niet-gepersonaliseerde techniek, zoals beschreven in \autoref{sec:chapt2_non_persionalised}. Dit hybride model zorgt dan voor een vloeiende overgang van niet-gepersonaliseerde technieken zolang er te weinig gebruikersgegevens zijn, tot volledig gepersonaliseerde aanbevelingen eens de voorkeuren van de gebruiker gekend zijn. Een alternatieve aanpak is de gebruiker expliciet vragen om zijn voorkeuren in een korte enquête. Hoewel dit ervoor zorgt dat de aanbevelingen vanaf het begin gepersonaliseerd zijn, gaat het afnemen wel ten koste van de gebruikerservaring (UX).

Aanbevelingssystemen kunnen helpen om gebruikers kennis te laten maken met dikkestaart-items. Echter stelt Fleder et al. \cite{recsys_diversity} dat doordat CF-algoritmen producten aanraadt op basis van consumpties en reviews, ze niet om kunnen met producten met beperkte beschikbare data. Hierdoor kan een Mattheüs-effect optreden waarbij populaire items nog populairder worden en onbekende items nooit aanbevolen worden. Het is dus belangrijk om het effect van het Cold-Startprobleem te minimaliseren.

\subsubsection{Datakwaliteit}
Het spreekt voor zich dat hoe preciezer de gebruikers en producten beschreven staan in de data, hoe makkelijker het is om correcte conclusies te trekken. Echter zijn niet alle algoritmen hier even gevoelig voor: Content-Based Filtering baseert zich enkel op de labels die bij de producten staan om aanbevelingen te maken. De correctheid, consistentie en precisie van deze labels is dus uitermate belangrijk voor CB. Om een dataset te laten voldoen aan deze eigenschappen is een significante investering nodig. Bij sommige datasets is het zelfs niet mogelijk om de items te verdelen in groepen en categorieën. Dit was de reden waarom in 1992 de eerste Collaborative Filtering-methode werd ontwikkeld. \cite{UUCF_original_paper} Bij machine learning-algoritmen is de gevoeligheid aan datakwaliteit implementatieafhankelijk. In tegenstelling tot Wide \& Deep Learning, verwacht DeepCoNN geen gelabelde dataset. De performantie van DeepCoNN blijft wel verbonden aan de kwaliteit van de ongestructureerde data: de geschreven reviews.
\subsubsection{Grootte dataset}
De performantie van machine learning-technieken schaalt logaritmisch met de grootte van de dataset. \cite{dataset_size_for_deep_learning} Het is dus belangrijk voor deze technieken om een zo groot mogelijke dataset te verzamelen zodat het model voldoende getraind kan worden.

Bij CB en CF is het niet nodig om een model te trainen. Deze zijn dus minder gevoelig aan de grootte van de dataset. Merk wel op er een schaarsheidprobleem kan optreden bij UUCF: als er veel items zijn, en gebruikers geven weinig feedback over deze items, dan is het mogelijk dat sommige gebruikers geen buren vinden of dat deze buren het doelitem nog niet beoordeeld hebben. \cite{cursus_hs9}
\subsubsection{Contextspecifiek}
De context is de combinatie van de dataset en het domein met diens specifieke eisen voor aanbevelingen. Een contextspecifieke techniek is een techniek waarbij een andere configuratie noodzakelijk is bij een wissel van context. Zo is in het domein van muziekaanbevelingen vaak de bedoeling om variëteit aan te brengen, zonder een scherpe verandering van genre/mood. Bij webwinkels is het dan weer anders: als een gebruiker daar een nieuwe laptop zoekt, zal een aanbevelingssysteem bijvoorbeeld alternatieven tonen die zo dicht mogelijk aansluiten bij de huidige keuze.

Traditionele methoden zijn weinig contextspecifiek. Er zijn weinig parameters (zoals het aantal buren in UUCF) om te optimaliseren. De gebruikte formules hebben slechts enkele varianten, zoals beschreven in \autoref{sec:chapt2_traditionele_methoden}. Dit staat lijnrecht tegenover de machine learning-technieken. Om de hoogste performantie te halen bij deze technieken is het noodzakelijk het effect van alle hyperparameters goed te begrijpen. DeepCoNN heeft bijvoorbeeld 14 hyperparameters die samen de volledige architectuur bepalen. \cite{deepconn_github}
\subsubsection{Explainability}
De explainability, of 'uitlegbaarheid' van een techniek is de mogelijkheid om te verklaren waarom die techniek een specifiek item aan een specifieke gebruiker heeft aanbevolen. Doordat de formules bij de traditionele methoden gekend zijn, is het quasi triviaal om dit te achterhalen. Zo kan men bij een UUCF-aanbevelingssysteem de buren van een gebruiker opvragen en zo uitrekenen waarom een item aanbevolen werd. Opnieuw staat dit lijnrecht tegenover de machine learning-methoden, waarbij zeker de technieken die gebruik maken van neurale netwerken beschreven worden als 'black box'. Het is mogelijk explainability in te bouwen in deze modellen, maar dit gaat ten koste van precisie. \cite{explainable_ai_recsys, explainable_recsys_autoencoders}

Explainability is belangrijk om gebruikers vertrouwen te laten hebben in het systeem. Zonder vertrouwen zal de gebruiker de aanbevelingen negeren. Dit kan een directe impact hebben op KPI van de diensten die men aanbiedt: als de gebruiker het systeem kan vertrouwen verhoogt de user experience en zal de gebruiker de dienst meer/langer gebruiken. Zo voorspelde een aanbevelingssysteem van Target (Amerikaanse warenhuiswinkelketen) dat een tienermeisje zwanger was. De vader reageerde hierop met 'Are you trying to encourage her to get pregnant?'. Het aanbevelingssysteem zag dat de dochter veel geurloze lotion kocht, wat typisch is voor zwangere vrouwen. Hierdoor bood het systeem meer artikels aan die zwangere vrouwen vaak kopen. \cite{recsys_baby_lotion_target} Deze reactie zou kunnen vermeden zijn, moest er een uitleg bij de aanbevelingen aangeboden werden.

Als de gebruiker weet waarom een item aanbevolen wordt, kan die ook rechtstreeks waardevolle feedback geven aan het aanbevelingssysteem. Zo gaf onder andere YouTube recent de mogelijkheid om diens aanbevelingen rechtstreeks te beïnvloeden door items te verwijderen uit de feed en feedback te geven waarom. \cite{youtube_on_recommendations} Deze feedback wordt dan gebruikt om de nieuwe aanbevelingen nog beter te kunnen personaliseren.

\mijnfiguur[H]{width=12cm}{fig/chapt2/youtube_recs.png}{Gebruikersfeedback op een aanbeveling op YouTube}{fig:chapt2_youtube_feedback_recsys}

\subsubsection{Diversiteit}
Fleder et al. \cite{recsys_diversity} stelt dat het gebruik van aanbevelingssystemen kan zorgen voor een toename van diversiteit op individueel niveau, maar een daling in de geaggregeerde diversiteit. Hoofdzakelijk algoritmen die zich baseren op labels, zoals CB filtering, zijn hier vatbaar voor. Deze algoritmen kunnen een 'echo-kamer' maken doordat steeds items met vergelijkbare labels worden aangeraden. Indien deze items geconsumeerd worden, wordt het gebruikersprofiel nog verder in die trend versterkt en ontstaat er een feedbackloop. Er zijn gevallen bekend waar gebruikers van YouTube geradicaliseerd zijn door het aanbevelingssysteem dat steeds extremere video's aanbiedt.

\cite{youtube_radicalisation} Onderzoek toont dat consumenten services als Spotify en Apple Music gebruiken om nieuwe muziek te leren kennen en daarvoor vertrouwen op aanbevelingssystemen. \cite{recsys_serendipity_music} Muzieksmaak evolueert per gebruiker verschillend. Providers moeten daarom ook proberen 'serendipity' te introduceren in hun aanbevelingen: nieuwe items die ver liggen van het gebruikersprofiel maar dat toch positief ontvangen worden. Het is echter niet triviaal om dergelijke items te voorspellen zonder vertrouwen te verliezen van de gebruiker: de 'serendipity' van een item meten werkt het best met (dure) expliciete feedback van een gebruiker.

Een oplossing hiervoor is willekeurige items toevoegen aan de aanbevelingen of expliciete feeds maken voor 'nieuwe' content. \cite{youtube_randomness, youtube_new_to_you} Zo heeft Spotify een 'Discover Weekly' playlist met deels nieuwe muziek voor de gebruiker en heeft YouTube een tabblad met 'New to you' video's over nieuwe onderwerpen (\autoref{fig:chapt2_youtube_new_to_you}).

\mijnfiguur[H]{width=5cm}{fig/chapt2/new_to_you_youtube.png}{'New to you' feed op YouTube}{fig:chapt2_youtube_new_to_you}

% TODO BRONNEN
% https://www.techtarget.com/searchcio/definition/transfer-learning
% https://www.datacamp.com/tutorial/an-introduction-to-using-transformers-and-hugging-face

\section{Natural Language Processing}
% todo thesis bert
In het gebied van NLP zal de computer de menselijk taal proberen te beheersen. Idealiter  kan een computer de taal begrijpen, verwerken en vervolgens correct genereren. NLP is toepasbaar in meerdere gebieden zoals vertalen, sentiment analysis, teksten samenvatten, spraakherkenning, etc.

\mijnfiguur[H]{width=12cm}{fig/chapt2/trend_NLP.png}{Stijgend aantal publicaties over Natural Language Processing \cite{NLP_popularity}}{fig:chapt2_research_trend_NLP}

Een sterke groei is duidelijk aanwezig binnen het gebied van NLP, dit is zichtbaar in \autoref{fig:chapt2_research_trend_NLP}. Dit komt onder andere door enkele recente ontdekkingen zoals bijvoorbeeld transformers (2017) en chatGPT (2022). Deze vooruitgang is ook een gevolg van de verbeteringen in het gebied van machine learning, zoals neurale netwerken en deep learning.

\subsection{Basistechnieken}
%todo basics such as preprocessing, splitting, ...
In deze sectie zullen we enkele TODO

\subsection{Transformers}
% todo grafiek dat het outperformed?
Transformers zijn een type neurale netwerken, ze werden voor het eerst geïntroduceerd in 2017 via de paper "Attention Is All You Need"\cite{attention_is_all_you_need}.
De modellen werden initieel gebruikt binnen het gebied van NLP om Engelse teksten te vertalen naar onder andere Duits en Frans. Nu zijn ze opgenomen als state of the art en worden ze gebruikt in diverse NLP taken.

% todo REF Recurrent neural networks, long short-term memory [13] and gated recurrent [7] neural networks
% https://www.datacamp.com/tutorial/an-introduction-to-using-transformers-and-hugging-face
% https://blog.knoldus.com/what-are-transformers-in-nlp-and-its-advantages

\subsubsection{Het netwerk}
De transformer neurale netwerken hebben een encoder-decoder structuur gebaseerd op het self-attention mechanisme. Aangezien het verwerken van woorden of tokens parallel kan, zal dit een significante verbetering in performantie geven. In definitie \ref{def:chapt2_transformers_encoder_decoder} gaan we van input sequentie X naar hidden representatie Y tot uiteindelijke output sequentie Z.

\begin{equation}
\begin{split}
X = [x_1, x_2, ..., x_n]  \\
\Downarrow \;\;\;\;\;\;\;\;\;\;\;\;\;\;\; \\
Y = [y_1, y_2, ..., y_n] \\
\Downarrow \;\;\;\;\;\;\;\;\;\;\;\;\;\;\; \\
Z = [z_1, z_2, ..., z_n]
\end{split}
\end{equation}
\label{def:chapt2_transformers_encoder_decoder}    


Het encoder-decoder gedeelte bestaat uit twee delen. Het eerste deel is het encoder gedeelte, met deze stap zullen we een gegeven input X opzetten in een hidden representatie Y. Dit zal gebeuren door de betekenis van woorden (of tokens) in de input X te encoderen gebaseerd op een approximatie van het belang van deze woorden (of tokens). Het verkrijgen van deze informatie zal gebeuren door meerdere lagen. Elke encoding laag bestaat uit een multi-head self-attention mechanisme gevolgd door een fully connected feed-forward laag.     

In bijhorend voorbeeld \ref{verb:chapt2_encoding_diff} zal de encodering van het vetgedrukt woord `het` significant veranderen. Dit komt omdat het belang van het woord waar `het` naar refereert, anders gezegd de woorden `glas` in de eerste zin en `kan` in de tweede zin, gewijzigd is.

\begin{Verbatim}[commandchars=\\\{\}]
    Hij giet water van de kan in het \verbatimbold{glas} todat \verbatimbold{het} vol is.
    Hij giet water van de \verbatimbold{kan} in het glas todat \verbatimbold{het} leeg is.
\end{Verbatim}
\label{verb:chapt2_encoding_diff}    

Het laatste gedeelte is de decoder. Deze zal de hidden representatie Y omzetten naar de output sequentie Z. Deze bestaat net zoals de encoder uit meerdere lagen. Een decoding laag zal bestaat uit een masked multi-head self-attention mechanisme gevolgd door multi-head self-attention die de encoding als input gebruikt. Ten slotte volgt nog een fully connected feed-forward laag.


\mijnfiguur[H]{width=12cm}{fig/chapt2/transformer_network_layout.jpg}{Encoder-decoder architectuur van een transformer neuraal netwerk. (encoder: links, decoder: rechts)\cite{attention_is_all_you_need}}{fig:chapt2_transformer_network_layout} % todo cite https://arxiv.org/pdf/1706.03762.pdf

\subsubsection{Self-attention mechanisme}
%TODO
Door dit mechanisme kan het model, gebaseerd op de waarde van bepaalde woorden, verschillende gewichten geven aan bepaalde delen van de input. Het maakt gebruik van een query matrix Q, key matrix K en een value matrix V. Deze worden verkregen door een vermenigvuldiging van de input sequentie met leerbare gewichten (learnable weights). Merk op dat deze matrices eigenlijk bestaan uit N vectoren van een bepaalde lengte D.

Voor dit voorbeeld zullen we een 'scaled dot-product attention' berekenen, merk op dat dat er andere varianten van attention bestaan. Om de attention gewichten G te berekenen gebruiken we vergelijking \ref{eq:chap2_attention_gewichten} gebruiken. Hier zullen we eerste een matrix vermenigvuldiging (dot product) toepassen op de query en key matrix. Vervolgens zullen we een schaalfactor S toepassen (in de paper \cite{attention_is_all_you_need} wordt S gelijk gesteld aan de $\sqrt{D_k}$, waarbij $D_k$ de lengte van een vector in de key matrix is). Om de attention gewichten te bekomen wordt de softmax functie nog toegepast.

\begin{equation}
G = softmax(QK^T / S)
\label{eq:chap2_attention_gewichten}
\end{equation}

waarbij:
\begin{conditions}
Q & Query matrix \\
K^T & Getransponeerde key matrix \\
S & schaalfactor \\
\end{conditions}

Vervolgens worden de attention gewichten G gebruikt om een gewogen som van de key matrix V te nemen zoals in vergelijking \ref{eq:chap2_attention_output}. Deze gewogen som W is dan de output van attention mechanisme en zal dus vervolgens worden doorgegeven aan de fully connected feed-forward laag.

\begin{equation}
W = GV
\label{eq:chap2_attention_output}
\end{equation}


De uiteindelijke architectuur beschreven in \autoref{fig:chapt2_transformer_network_layout} maakt gebruik van multi-head attention. Deze zal het self-attention mechanisme in parallel uitvoeren met H groepen van kleinere matrices $Q_i$, $V_i$ en $K_i$ met $i=1,2,...,H$. Deze groepen worden verkregen door de originele matrices lineair te projecteren. De output van het attention mechanisme wordt ten slotte weer samengevoegd om zo de uiteindelijke output te bekomen. Dit proces wordt ook gevisualiseerd in \autoref{fig:chapt2_scaled_dot_multi_head}. Door het gebruik van multi-head attention zal het model sneller werken.

\mijnfiguur[H]{width=12cm}{fig/chapt2/scaled_dot_and_multi_head_attention.jpg}{Visualisatie van scaled dot-product attention (links). Visualisatie van multi-head attention met h attention lagen op basis van scaled dot-product attention (rechts)\cite{attention_is_all_you_need}}{fig:chapt2_scaled_dot_multi_head} % todo cite https://arxiv.org/pdf/1706.03762.pdf

\subsubsection{Trainen van een model}
Het trainen van een nieuw model is een lastige taak. Dit komt door de hoeveelheid vereiste data. Na het verzamelen van de data moet het model nog getrained worden, hiervoor is een grote hoeveelheid computationele kracht nodig.

In realiteit wordt het concept van transfer learning vaak toegepast. Dit is een techniek die eerst een model zal trainen op een gigantische dataset en vervolgens hetzelfde model finetunen met een kleinere dataset.

Binnen de context van NLP zullen modellen tijdens het trainen de relaties tussen de verschillende woorden en zinnen leren. Hierdoor kunnen ze de woorden uit een taal of zelfs meerdere talen geëncodeerd voorstellen. Vervolgens zal het model gefinetuned worden op een kleinere dataset voor één specifiek taak zoals bijvoorbeeld sentiment analysis. Indien we het model voor een nieuwe taak, zoals bijvoorbeeld vertalen, willen gebruiken zullen we het pre-trained LLM hergebruiken en finetunen met een andere dataset en taak. Dit concept is visueel voorgesteld in \autoref{fig:chapt2_transfer_learning_LLM}.

\mijnfiguur[H]{width=12cm}{fig/chapt2/transfer_learning_LLM.jpg}{Transfer learning waarbij een pre-trained LLM hergebruikt wordt voor verschillende taken}{fig:chapt2_transfer_learning_LLM} 

\subsection{BERT}
% todo ref https://www.techtarget.com/searchenterpriseai/definition/BERT-language-model
% todo ref https://arxiv.org/pdf/1810.04805.pdf
BERT, wat staat voor Bidirectional Encoder Representations from Transformers, is een open source deep learning model gebaseerd op transformers gecreëerd door Google.

Het is een pre-trained model, getrained op grote hoeveelheden ongelabelde tekstuele data zonder specifieke taak. Hierdoor kan het principe van transfer learning toegepast worden. Daarom kan BERT gebruikt worden voor verschillende taken zoals vertalen, sentiment analysis, teksten samenvatten, etc.

Wat BERT onderscheid van vorige transformer modellen is zijn begrip van de context waarin een woord gebruikt wordt. Dit komt omdat BERT gebruikt maakt van een bidirectionele transformer.  Dat wil zeggen dat elke input sequentie in beide richtingen zal verwerken worden, met andere woorden zowel van links naar rechts als van rechts naar links.


% todo enkele voorbeelden van wat we met NLP kunnen doen
\subsection{Topic modelling}

Topic modelling binnen NLP is een techniek om uit verschillende tekstuele documenten verborgen onderwerpen te halen. Deze onderwerpen noemen topics en bestaan uit meerdere woorden die semantisch dicht bij elkaar liggen. Een voorbeeld van een topic die dieren voorstelt is onderstaande verzameling van woorden.

\[
topic\_n = \{leeuw, aap, varken, koe, olifant, kat, hond, goudvis\}
\]

Merk op dat deze verzameling niet uniek is en sterkt afhangt van meerdere factoren zoals onder andere de trainingsdata en het aantal topics. Indien de trainingsdata verschillende documenten over huisdieren en wilde dieren bevat zal de verdeling anders zijn. Een mogelijke verdeling voor respectievelijk huisdieren tegenover wilde dieren is hieronder te vinden.

\[
topic\_1 = \{kat, hond, goudvis\}
\]
\[
topic\_2 =\{leeuw, poema, olifant\}
\]

Deze machine learning algoritmen zijn unsupervised en hebben dus geen gelabelde trainingsdata nodig. Er is dus geen enkele voorkennis over de onderwerpen nodig, dit geldt voor zowel de documenten als de topics zelf. 

Topic modelling kan op diverse manieren gebruikt worden. Een voorbeeld is om beoordelingen van klanten te analyseren. Indien men kan identificeren waarom een bepaalde klant een positieve of negatieve score achterlaat kan een bedrijf doelgericht hun diensten of producten verbeteren. Een andere toepassing is het maken van aanbevelingen. In dit geval kan men de teksten, gelezen door een bepaalde gebruiken, analyseren en vervolgens gelijkaardige teksten aanbevelen.

\subsection{BERTopic}

BERTopic is een krachtig topic modelling algoritme, het maakt gebruik van onder andere clustering en dimensionaliteit reductie. Het algoritme werd BERTopic genoemd aangezien de tekstuele data initieel werd omgezet in vectoren aan de hand van BERT, nu zijn er diverse mogelijkheden om deze stap te voltooien. Hierdoor zal de contextuele informatie ook verwerkt worden en vervolgens ook omvat zijn in de uiteindelijk topics. Dit is een van de grootste voordelen tegenover andere topic modelling algoritmen zoals LDA (Latent Dirichlet Allocation).

\mijnfiguur[H]{width=12cm}{fig/chapt2/bertopic_algo.jpg}{\cite{todo bertopic}}{fig:chapt2_bertopic_algo} % https://maartengr.github.io/BERTopic/algorithm/algorithm.html

De opbouw van BERTopic bestaat uit 6 stappen zoals aangetoond in \autoref{fig:chapt2_bertopic_algo}, hiervan is de laatste stap optioneel is. In de volgende secties zullen bespreken wat de stappen inhouden, inclusief mogelijk implementaties.

\subsubsection{Genereren van een embedding}

\subsubsection{Dimensionaliteit reductie}

\subsubsection{Clustering}



\chapter{Dataset}
\label{sec:chapt3}

Voor ons onderzoek beroepen we ons op de Yelp Dataset. Dit is een open dataset met verschillende voordelen: de beschikbare data wordt geleverd met een open licentie die het gebruik voor onderzoek toestaat. \cite{Yelp_Dataset} De data is een verzameling van reviews van echte gebruikers over echte restaurants, en niet synthetisch uitgebreid of aangepast. Hierdoor is de kans groter dat de resultaten van het onderzoek in de praktijk ook relevant zijn. De dataset omvat veel meer dan enkel restaurants. Allerhande soorten bedrijven zijn aanwezig, van juweelwinkels tot autogarages. In de volgende onderdelen gaan we enkel verder met bedrijven die tot de categorie 'restaurant' of 'food trucks' behoren. 

\section{Eigenschappen}
\label{sub:chapt3_eigenschappen_dataset}
De verzamelde data beschrijft 153 346 bedrijven uit 11 steden in De Verenigde Staten van Amerika en Canada. 1 987 897 gebruikers gaven 6 990 280 reviews over deze bedrijven. Ons onderzoek spitst zich toe op restaurants. Na filteren blijven 52 533 restaurants over, of 34\%. Het aantal reviews daalt minder sterk: 68\% blijft over, dus 4 731 031.

\subsection{Gebruikers}
\subsubsection{Reviews per gebruiker}
\label{sec:chapt3_reviews_per_gebruiker}
Iedere gebruiker heeft gemiddeld 3,27 reviews, waarbij 82\% minder dan het gemiddelde aantal reviews heeft. De mediaan is ligt maar op $1$ review per gebruiker. Door het Cold-Startprobleem (\ref{sec:chapt2_cold_start}) zal het voor deze gebruikers moeilijker worden om accurate aanbevelingen te maken. In \autoref{fig:chapt3_verdeling_aantal_reviews_per_gebruiker} zijn gebruikers met meer meer dan 20 reviews zijn weggelaten.

Een review bestaat uit gemiddeld uit 7,5 zinnen wat in totaal voor ongeveer 36 miljoen zinnen of documenten zal zorgen. De meeste reviews bestaan uit 3 tot 6 zinnen, zoals afgebeeld in \autoref{fig:chapt3_verdeling_aantal_zinnen_per_review}.

\begin{figure}[H]
    \begin{subfigure}{.5\textwidth}
        \centering
        \includegraphics[width=1\linewidth]{fig/chapt3/verdeling_aantal_reviews_per_gebruiker.png}
        \caption{Histogram aantal reviews per gebruiker}
        \label{fig:chapt3_verdeling_aantal_reviews_per_gebruiker}
    \end{subfigure}
    \begin{subfigure}{.5\textwidth}
        \centering
        \includegraphics[width=1\linewidth]{fig/chapt3/zin_per_review.png}
        \caption{Histogram aantal zinnen per review}
        \label{fig:chapt3_verdeling_aantal_zinnen_per_review}
    \end{subfigure}
    \caption{Hoeveelheid beschikbare data per gebruiker}
\end{figure}

\subsubsection{Verdeling scores}
De mediaan score over alle gebruikers is 4 van de 5 sterren. De hogere scores komen vaker voor, met 5 de individueel meest voorkomende score.
\begin{figure}[H]
    \begin{subfigure}{.5\textwidth}
        \centering
        \includegraphics[width=1\linewidth]{fig/chapt3/verdeling_mediaan_scores_per_gebruiker.png}
        \caption{Histogram mediaan score van reviews per gebruiker}
        \label{fig:chapt3_verdeling_mediaan_scores_per_gebruiker}
    \end{subfigure}
    \begin{subfigure}{.5\textwidth}
        \centering
        \includegraphics[width=1\linewidth]{fig/chapt3/verdeling_alle_scores.png}
        \caption{Histogram score van alle individuele reviews}
        \label{fig:chapt3_verdeling_alle_scores}
    \end{subfigure}
    \caption{Histogrammen over scores}
\end{figure}
We moeten bij de evaluatie van ons aanbevelingssysteem dus rekening houden met deze non-uniforme verdeling. De klassen die lagere scores voorstellen zijn minder vertegenwoordigd in de data, om de gebruikerservaring hoog te houden moeten we voorkomen dat slechte items toch aanbevolen worden.

\subsubsection{Trends binnen dezelfde gebruiker}
De afstand tussen de minimum- en maximumscore die éénzelfde gebruiker geeft is vaak vrij groot. Dit wil zeggen dat gebruikers het volledige spectrum aan scores gebruiken om hun mening uit te drukken. Zelfs wanneer we de meest extra waarden wegfilteren, blijft deze conclusie gelden. Er zijn dus weinig gebruikers die steeds dezelfde score geven. In \autoref{fig:chapt3_verdeling_score_min_max_combined} zijn gebruikers met minder dan 5 reviews weggelaten.
\begin{figure}[H]
    \begin{subfigure}{.5\textwidth}
        \centering
        \includegraphics[width=1\linewidth]{fig/chapt3/verdeling_score_min_max.png}
        \caption{Maximumscore - minimumscore}
        \label{fig:chapt3_verdeling_score_min_max}
    \end{subfigure}
    \begin{subfigure}{.5\textwidth}
        \centering
        \includegraphics[width=1\linewidth]{fig/chapt3/verdeling_score_min_max_percentiel.png}
        \caption{$95^e$ percentiel - $5^e$ percentiel}
        \label{fig:chapt3_verdeling_score_min_max_percentiel}
    \end{subfigure}
    \caption{Histogrammen over verschil tussen hoge en lage scores binnen dezelfde gebruiker}
    \label{fig:chapt3_verdeling_score_min_max_combined}
\end{figure}

\subsection{Restaurants}
\subsubsection{Reviews per restaurant}
Ieder restaurant uit de dataset heeft minstens 5 reviews van gebruikers. In \autoref{fig:chapt3_verdeling_aantal_reviews_per_restaurant} zijn restaurants met meer dan 150 reviews weggelaten om de Figuur leesbaar te houden. In werkelijkheid gaat de asymptotische trend verder tot en met maximaal $7673$ reviews (\autoref{fig:chapt3_boxplot_aantal_reviews_per_restaurant}. Merk op dat de dataset veel minder sparse is als we groeperen per restaurant (\ref{sec:chapt3_reviews_per_gebruiker}): weinig gebruikers hebben 5 of meer reviews geschreven, maar ieder restaurant heeft wel minstens 5 reviews. Deze schaarsheid aan reviews per gebruiker suggereert dat IICF beter zal werken op deze dataset dan UUCF. \cite{cursus_hs9}


\begin{figure}[H]
    \begin{subfigure}{.7\textwidth}
        \centering
        \includegraphics[width=1\linewidth]{fig/chapt3/verdeling_aantal_reviews_per_restaurant.png}
        \caption{Histogram (tot 150 reviews)}
        \label{fig:chapt3_verdeling_aantal_reviews_per_restaurant}
    \end{subfigure}
    \begin{subfigure}{.25\textwidth}
        \centering
        \includegraphics[width=1\linewidth]{fig/chapt3/boxplot_aantal_reviews_per_restaurant.png}
        \caption{Boxplot (volledig)}
        \label{fig:chapt3_boxplot_aantal_reviews_per_restaurant}
    \end{subfigure}
    \caption{Aantal reviews per restaurant}
    \label{fig:chapt3_aantal_reviews_per_restauant_combined}
\end{figure}



\subsubsection{Verdeling scores}
De meeste restaurants hebben een mediaan van 4 op 5 sterren. Dit komt op het eerste zicht niet overeen met de bevinding dat de meeste gebruikers een mediaan van 5 op 5 sterren hebben voor hun reviews. Echter blijkt dat de grootste groep van gebruikers met een mediaan van 5 op 5 maar 1 of 2 reviews hebben achtergelaten. We concluderen hieruit dat gebruikers die een groter aantal reviews achterlaten een lagere mediaan hebben.
\mijnfiguur[H]{width=10cm}{fig/chapt3/verdeling_mediaan_scores_per_restaurant.png}{Histogram mediaan reviews per restaurant}{fig:chapt3_verdeling_mediaan_scores_per_restaurant}

\subsubsection{Trends binnen hetzelfde restaurant}
Het verschil tussen de hoogste en laagste score van een restaurant is vaak extreem. Zelfs zonder de 10\% meest extreme scores blijft er een groot verschil. We kunnen hieruit concluderen dat er zeer weinig restaurants bestaan waarvoor alle gebruikers het eens zijn over de score. Dit toont dus aan dat er nood is aan gepersonaliseerde aanbevelingen.
\begin{figure}[H]
    \begin{subfigure}{.5\textwidth}
        \centering
        \includegraphics[width=1\linewidth]{fig/chapt3/verdeling_score_min_max_restaurant.png}
        \caption{Maximumscore - minimumscore}
        \label{fig:chapt3_verdeling_score_min_max_restaurant}
    \end{subfigure}
    \begin{subfigure}{.5\textwidth}
        \centering
        \includegraphics[width=1\linewidth]{fig/chapt3/verdeling_score_min_max_percentiel_restaurant.png}
        \caption{$95^e$ percentiel - $5^e$ percentiel}
        \label{fig:chapt3_verdeling_score_min_max_percentiel_restaurant}
    \end{subfigure}
    \caption{Histogrammen over verschil tussen hoge en lage scores binnen hetzelfde restaurant}
    \label{fig:chapt3_verdeling_score_min_max_combined_restaurants}
\end{figure}

\subsubsection{Attributen en categorieën}
\label{sec:chapt3_labeled_data_attributes_categories}
Een bedrijf dat de categorie 'restaurant' bevat, zal vaak nog andere categorieën bevatten die het type restaurant beter beschrijven, zoals 'fastfood' of 'Italian'. Deze categorieën kunnen dus gebruikt worden om de restaurants te beschrijven aan de hand van labels. Er zijn in totaal 1311 verschillende categorieën die de restaurants modelleren. Dit grote aantal zorgt ervoor dat ieder restaurant vrij uniek beschreven is, wat zorgt voor een ijle dataset. Hierdoor wordt het ook moeilijker om gelijkaardige items te vinden op basis van labels. Een restaurant bevat gemiddeld 4,25 categorieën (\autoref{fig:chapt3_verdeling_aantal_categorieën_per_restaurant}).\newline
Als een gebruiker een review achterlaat over een restaurant dan associëren we die gebruiker met de categorieën van dat restaurant. Zo zien we dat de categorieën van een gebruiker ook vrij lineair schalen met het aantal reviews (\autoref{fig:chapt3_stijging_categorieen_per_gebruiker}). Dit toont dat een gebruiker vaak verschillende types restaurants bezoekt. Dit suggereert dat Content-Based filtering niet goed zal werken op deze dataset.


\begin{figure}[H]
    \begin{subfigure}{.5\textwidth}
        \centering
        \includegraphics[width=1\linewidth]{fig/chapt3/verdeling_aantal_categorieën_per_restaurant.png}
        \caption{Histogram aantal categorieën per restaurant \\ \\}
        \label{fig:chapt3_verdeling_aantal_categorieën_per_restaurant}
    \end{subfigure}
    \begin{subfigure}{.5\textwidth}
        \centering
        \includegraphics[width=1\linewidth]{fig/chapt3/stijging_categorieen_per_gebruiker.png}
        \caption{Aantal unieke categorieën geassocieerd met een gebruiker stijgt lineair met het aantal reviews}
        \label{fig:chapt3_stijging_categorieen_per_gebruiker}
    \end{subfigure}
    \caption{Analyse categorieën bij restaurants}
    \label{fig:chapt3_categorieen_combined}
\end{figure}

Een attribuut uit de Yelp dataset is vaak een faciliteit die een restaurant bezit, zoals parking of de mogelijkheid tot betalen met kredietkaart. We houden bij het maken van aanbevelingen geen rekening met de meeste van dit soort attributen: de verwachte use case is dat een gebruiker een manuele filter gebruikt om dergelijke eisen te verwerken. Dit type attributen zullen we dus niet gebruiken. Dezelfde redenering gaat op voor de locatie van een restaurant. We verwachten dat deze data gebruikt wordt om restaurants te filteren die fysiek dicht bij de gebruiker zijn, voordat de verwachte score wordt berekend.\newline
Er zijn wel enkele nuttige attributen zoals 'atmosphere', dat de sfeer van een restaurant beschrijft in één woord zoals 'romantic' of 'classy', en 'priceRange', wat een restaurant opdeelt in één van vier prijsklassen. In tegenstelling tot categorieën, zijn attributen niet consistent aanwezig bij ieder restaurant. 

\chapter{Experimenten}

\section{Voorgestelde architectuur}
In deze thesis onderzoeken we of de combinatie van tekstuele data aan de hand van transformermodellen kan omgezet worden in features, die het voorspellingsvermogen van een neuraal netwerk positief beïnvloeden. Ons basisidee ziet er uit zoals beschreven in \autoref{fig:chapt4_architectuur_begin}: eerst worden de geschreven reviews door een transformermodel omgezet naar numerieke features. Deze worden dan toegevoegd aan de input van een neuraal netwerk. We maken dus gebruik van een 'feature augmentation'  hybride model (\ref{sec:chapt2_hybride_modellen}) met machine learning-technieken (\ref{sec:chapt2_machine_learning_modellen}). De overige features voor het neurale netwerk komen uit de dataset, eventueel verwerkt met feature engineering. Dit betreft dan features die de restaurants beschrijven, zoals het type restaurant (bvb "fastfood"). In \autoref{sec:chapt4_tekst_naar_features} beschrijven we welke combinatie van technieken het beste werkt om de geschreven reviews om te zetten naar features voor het neurale netwerk. In \autoref{sec:chapt4_neuraal_netwerk} doen we onderzoek naar de optimale vorm van het neurale netwerk om de scores zo precies mogelijk te kunnen voorspellen.

\mijnfiguur[H]{width=12cm}{fig/chapt4/predictor/architectuur_begin.png}{Schets van de initiële architectuur}{fig:chapt4_architectuur_begin}
% TODO: throwback naar chapt2 waarin ik de uitdagingen uitleg en geef aan in welke categorie we zitten

\section{Dataflow}
\label{sec:chapt4_data_flow}
Zoals aangeduid in \autoref{fig:chapt4_architectuur_begin}, verwacht het neurale netwerk vier vectoren als inputdata:
\begin{itemize}
    \item Gebruikersprofiel gecreëerd door NLP (geschreven reviews)
    \item Restaurantprofiel gecreëerd door NLP (geschreven reviews)
    \item Gebruikersprofiel gebaseerd op labels (Yelp dataset labels)
    \item Restaurantprofiel gebaseerd op labels (Yelp dataset labels)
\end{itemize}

Het neuraal netwerk gebruikt deze data om een voorspelling te genereren voor de score die de gespecificeerde gebruiker aan het gespecificeerde restaurant geeft.

\autoref{fig:chapt4_data_flow} toont welke data wordt verwerkt tot profielen, gebruikt wordt voor trainen en voor testen. Bij machine learning-technieken is het afgetekend en correct opsplitsen van data in train- en testset uitermate belangrijk. Dit is nodig om de generaliteit van het model te garanderen en overfitting te voorkomen. Door de complexe architectuur van het voorgesteld netwerk is de dataflow niet evident:

\begin{enumerate}
    \item Eerst splitsen we de gebruikers uit de dataset op in twee groepen, steeds met bijhorende reviews van die gebruikers. De eerste groep stelt de trainset voor, en omvat 80\% van de volledige dataset. De andere groep is de testset, met de overige 20\% van de data.
    \item We gaan eerst verder met de trainset. Het neurale netwerk verwacht vier vectoren, die steeds een gebruikersprofiel of restaurantprofiel voorstellen. Echter kunnen we niet alle reviews van een gebruiker verwerken om deze profielen op te stellen! 
    
    Indien we dat wel zouden doen, zouden de gegevens van die review verwerkt worden in het profiel. Als we later het model trainen op deze review, bevat de profielvector gegevens over de gebruiker die we op dat punt nog niet zouden mogen weten. We zouden dus trainen op data die in een echte deployment nog niet beschikbaar is.

    We splitsen daarom de trainset nogmaals op in twee delen door willekeurig reviews te samplen. De eerste set stelt dan de geschiedenis voor van een gebruiker of restaurant, en wordt strikt gebruikt om gebruikers- en restaurantprofielen op te stellen. Dit omvat 70\% van de reviews uit de trainset.\newline
    De overige 30\% van de reviews wordt gebruikt om het neurale netwerk te trainen: we linken de gebruiker en restaurant van die reviews met de bijhorende profielen gemaakt uit het eerste deel. Dit stelt de input van het neurale netwerk voor. De score van die reviews uit het tweede deel stelt de output voor. Hiermee hebben we dus alle informatie om het neurale netwerk te trainen. 

    \item Na het trainen van neuraal netwerk, wordt analoog aan stap 2 ook de testset verwerkt tot profielen en ongeziene reviews. Deze worden dan gebruikt om te meten hoe goed het model in staat is om aan de hand van de profielen een score te voorspellen.

    \item Er wordt enkele keren met dezelfde profielen en dezelfde ongeziene reviews getraind. Doordat de profielen niet steeds opnieuw moeten uitgerekend worden, gaat het verwerken van deze ``sub-epochs'' sneller.

    \item Na iedere (volwaardige) epoch worden dezelfde train- en testset opnieuw opgedeeld tot data voor profielen en ongeziene reviews. Hierdoor kunnen we de volledige dataset efficiënter benuttigen.
\end{enumerate}


\mijnfiguur[H]{width=17cm}{fig/chapt4/predictor/data_flow.png}{Schets van dataflow doorheen het model}{fig:chapt4_data_flow}

% todo we use Azure cluster for offline BERT, trust me :) @arno
\section{Tekst naar features}
\label{sec:chapt4_tekst_naar_features}
In deze sectie zullen we beschrijven hoe we uit de tekstuele reviews features (profielen) zullen verkrijgen. Ook beschrijven we de redeneringen achter de algoritmen, met andere woorden wat deze features moeten voorstellen. We zullen deze algoritmen vooral baseren op BERTopic. Aangezien we verschillende modellen zullen verkrijgen moeten we deze zo objectief mogelijk evalueren, hoe we dit doen staat beschreven in \autoref{sub:chapt4_testsetup}. BERTopic zal van de reviews een clustering maken, dit volstaat nog niet als features voor het neurale netwerk. We zullen van deze clustering eerst nog gebruikers- en restaurantprofielen moeten maken. Dit proces is ook gevisualiseerd in \autoref{fig:chapt4_structuur_evaluatie_bertopic}.

\mijnfiguur[H]{width=16cm}{fig/chapt4/NLP/structuur_evaluatie_bertopic.jpg}{Visualisatie van het proces om tekstuele reviews om te zetten in de uiteindelijk gebruikers- en restaurantprofielen.}{fig:chapt4_structuur_evaluatie_bertopic}

\subsection{Testset-up}
\label{sub:chapt4_testsetup}
Zoals gevisualiseerd in \autoref{fig:chapt4_structuur_evaluatie_bertopic} kunnen we ons algoritme op meerdere plaatsen evalueren. In volgende paragrafen zullen we beide manieren met elkaar vergelijken. We zullen onder andere de werking beschrijven, enkele voor- en nadelen oplijsten en een conclusie trekken.

\subsubsection{Evaluatie van profielen}

Een eerste mogelijkheid is om de uiteindelijke verkregen features te evalueren. In dit geval zullen dat de gebruikers- en restaurantprofielen zijn, zoals beschreven in \autoref{sub:chapt2_gebruikersprofielen}. Aan de ene kant modelleren ze wat een bepaalde gebruiker belangrijk vindt via zijn gebruikersprofiel. Aan de andere kant zullen we dit combineren met een restaurantprofiel, hiermee geven we de specialiteiten en andere eigenschappen van een bepaald restaurant weer. Om deze profielen te evalueren zullen we gebruik maken van de architectuur afgebeeld in \autoref{fig:chapt4_architectuur_begin}. We zullen het volledige neurale netwerk constant houden met uitzondering van deze profielen. Op deze manier kunnen we de profielen objectief beoordelen door de output van het neurale netwerk te evalueren. Ten slotte zullen we de (combinatie van) profielen waarvoor het model het beste presteert selecteren.

\subsubsection{Evaluatie van de clustering}

De tweede manier is aan de hand de verkregen clusters zoals beschreven in \autoref{sub:chapt2_bertopic_clustering}. Bij de keuze van een evaluatiemetriek voor deze clusters moeten we rekening houden met enkele aspecten. Het eerste is dat we geen gelabelde data hebben. In theorie kunnen we de zinnen handmatig labelen, helaas zal dit niet altijd even accuraat zijn. Hierbij komt ook nog het probleem dat elk BERTopic model verschillende topics zal maken, het gevolg is dat we handmatig gelabelde data niet kunnen hergebruiken. Met deze redenen zullen we metrieken die gebruik maken van de ground truth labels uitsluiten en gebruik maken van de technieken beschreven in \autoref{sec:chapt2_clustering_evaluation}.

Het doel van deze evaluatiemethode is om een idee te krijgen hoe goed de clustering werkt zonder eerst een volledig neuraal netwerk te trainen. Deze zullen vaak sneller beschikbaar zijn. De resultaten van de metrieken voor de verschillende modellen zullen we bespreken in \autoref{sub:chapt4_eval_clustering}. Ten slotte zullen we kijken of we een verband tussen de metrieken en de evaluatie van de profielen kunnen leggen in \autoref{sub:chapt4_compare_eval_methods}.

\subsection{Clustering via BERTopic}
In deze sectie zullen we bespreken hoe we BERTopic zullen gebruiken om een clustering te genereren. Deze modellen zullen we dan gebruiken bij het creëren van gebruikers- en restaurantprofielen in \autoref{sec:chapt4_nlp_profielen}. We hebben gebruik gemaakt van een bestaande implementaties van BERTopic \cite{bertopic_homepage} waarbij we de verschillende lagen aanpassen door onder andere een sentence-BERT te gebruiken van \cite{sentence_transformers_implementation}. Merk op dat het embedding model getraind is op algemene data. In het ideale geval hebben we een gelabelde dataset over de restaurants die kan aangeven hoe gelijkaardig twee zinnen zijn. Helaas is dit niet het geval waardoor finetunen van het sentence-BERT model niet mogelijk is.

% todo in elke titel van subsubsectie BERTOPIC?
\subsubsection{Standaard BERTopic}
Het initiële model zal gebruik maken van de standaardimplementatie, zoals beschreven in \autoref{sub:chapt2_bertopic}, in combinatie met sentence-BERT. Deze implementatie is voor de volledigheid gevisualiseerd in \autoref{fig:basismodel_bertopic}. Bovenop dit model zullen we ook nog een extra finetuning laag toevoegen, namelijk KeyBERTInsipired beschreven in \autoref{sub:topic_representatie}

\mijnfiguur[H]{width=5cm}{fig/chapt4/NLP/basismodel_bertopic.jpg}{Visualisatie van de standaardimplementatie van BERTopic \cite{bertopic_algo}.}{fig:basismodel_bertopic}

De eerste stap is het bepalen wat de documenten zullen voorstellen, hiervoor hebben we meerdere mogelijkheden. De meest voor de hand liggende mogelijkheid is dat we één review als een document beschouwen. Dit zal betekenen dat we ongeveer 4,7 miljoen documenten hebben zoals beschreven in \autoref{sub:chapt3_eigenschappen_dataset}. Helaas brengt deze aanpak meerdere problemen met zicht mee. 

Één probleem volgt rechtstreeks uit het gebruik van BERTopic, zoals beschreven in \autoref{sub:chapt2_bertopic} kunnen we namelijk een document maar aan één cluster toevoegen. Dit is tegenstrijdig met de praktijk, waar reviews meerdere onderwerpen aankaarten zoals lekker eten maar slechte service. Aangezien een cluster overeenkomt met precies één onderwerp is dit geen ideale match. We zullen deze complicatie deels vermijden door de reviews op te splitsen in zinnen. Dit zal gebeuren via een tokenizer zoals beschreven in \autoref{sub:chapt2_tokenization}. Hierbij veronderstellen we wel nog steeds dat één zin overeenkomt met één onderwerp. Gebaseerd op een steekproef lijkt dit voor de meeste zinnen uit de reviews wel het geval. Onze implementatie gebruikt de sentence tokenizer van SpaCy \cite{spacy_main}.

Via deze methode kunnen we aan elk van de 36 miljoen zinnen een cluster toekennen, hierdoor kunnen we meerdere onderwerpen aan één review toekennen. Voor de rest van deze masterthesis zullen we één document gelijkstellen aan één zin uit een review. Een bijkomend nadeel van deze methode is dat sommige zinnen uit geen enkel relevant onderwerp bestaan. Dit zijn zinnen die, ongerelateerd aan het restaurant, een verhaal vertellen of bepaalde omstandigheden omschrijven. Het gevolg kan waargenomen worden in de representatie van enkele topics, deze brengen geen waardevolle informatie. 

De volgende uitdaging is de schaalbaarheid van het algoritme. Zoals beschreven in vorige paragraaf hebben we ongeveer 36 miljoen documenten die we zullen moeten clusteren. Voor het genereren van de embeddings zal dit al snel een probleem geven wegens de grote hoeveelheid verreist geheugen. Deze hoeveelheid zal nog toenemen eens we een clustering zullen creëren. Voor ongeveer 2\% van de data (100 000 reviews gelijk aan 650 000-700 000 documenten) is dit nog mogelijk met een geheugen van 64GB. Vervolgens kunnen we de overige documenten bevragen aan de hand van het getrainde model.

\subsubsection{Online BERTopic}
Een online variant, ook wel incrementeel genoemd, is een algoritme dat gebruik kan maken van een datastroom zonder de volledige input te weten. Hierdoor kan het algoritme de data in kleinere delen verwerken, wat in ons geval interessant is. Een bijkomend voordeel hiervan is de mogelijkheid om toekomstige data efficiënt te verwerken. We zullen door het model kunnen updaten met nieuwe data zonder alles opnieuw te moeten trainen. In ons geval is dit niet relevant, maar dit is zeker een nuttige eigenschap voor een productieomgeving.

Om BERTopic om te zetten naar een online algoritme zullen we alle stappen moeten omzetten naar een online variant, indien dit nog niet het geval is. Hoe deze stappen omgezet worden staan hieronder beschreven, deze transformatie wordt ook gevisualiseerd in \autoref{fig:chapt4_bertopic_online_transformation}

% todo REF naar H2 indien we het daar uitleggen.
\begin{itemize}
    \item \textbf{De embedding} komt van een LLM, deze hoeft niet continue getraind te worden en kan al bevraagd worden voor ongeziene data. Hiervoor is er dus geen aanpassing nodig
    \item Voor \textbf{dimensionaliteitsreductie} gebruiken we een online variant, namelijk IPCA.
    \item Bij het \textbf{clusteringsalgoritme} schakelen we over naar een online K-Means algoritme genaamd MiniBatch K-Means.
    \item \textbf{De BOW representatie} wordt ook aangepast naar zijn online variant.
    \item Aangezien \textbf{c-TF-IDF} en verdere aanpassingen gebaseerd zijn op de BOW representatie zal deze online zijn indien het BOW algoritme online werkt.
\end{itemize}

\mijnfiguur[H]{width=12cm}{fig/chapt4/NLP/bertopic_to_online.jpg}{Transformatie van de standaard BERTopic structuur naar een online algoritme.}{fig:chapt4_bertopic_online_transformation}

Door het gebruik van deze structuur kunnen we het model op een grotere hoeveelheid data trainen. We hebben dus een schaalbaar model gecreëerd. De trainingstijd van dit model zal lineair stijgen op basis van het aantal trainingsdata. Als data kunnen we voor dit model de volledige dataset gebruiker, dit komt omdat we hier geen rechtstreekse features genereren voor het neurale netwerk.

\subsubsection{Guided BERTopic}
Een laatste experiment voor clustering maakt gebruik van guided BERTopic \cite{bertopic_guided}, een reeds geïmplementeerde variant op het standaard algoritme. Hierbij geven we per topic een lijst van woorden mee die het onderwerp van deze topic voorstellen. Deze lijst van woorden wordt de seed van de topic genoemd. Het uiteindelijke model houdt hiermee rekening en zal de kans vergroten om deze seeded topics als finale output te genereren. Merk op dat dit volgens de auteur niet altijd het geval is. Vaak zullen deze topics aangepast of opgesplitst worden, tenzij de ze extreem accuraat zijn.

Via deze methode willen we het probleem van irrelevante topics, als gevolg van het opsplitsen in zinnen, vermijden. We doen dit door naast de vaste lijst van topics, het model ruimte te geven om extra topics te genereren. Het doel is dat deze topics gevuld worden met de irrelevante zinnen.

Na enkele pogingen met verschillende parameters zien we telkens hetzelfde probleem opduiken. Door de grote hoeveelheid data zien we dat de voorgedefinieerde topics overspoeld worden met andere data. Hierdoor blijft er weinig over van de originele seeds. Een mogelijke oplossing is om minder data te gebruiken, hierdoor kunnen we ook gebruik maken van de standaardimplementatie van BERTopic. Door minder data te gebruiken zijn de resultaten significant slechter, dit zullen we uitgebreid bespreken in \autoref{sub:chapt4_nlp_resultaten}.

\subsubsection{Optimalisaties}
% cache => bijhouden
% append only => dus niet vaak herberekenen (enkel als model veranderd)
% schaalbaar => in praktijk ook nuttig
Aangezien we het model vaak zullen bevragen met dezelfde data is het mogelijk om deze op te slaan. Hierdoor kunnen deze later snel ingelezen worden voor hergebruik. Merk op dat reviews in normale omstandigheden niet wijzigen, er kunnen enkel nieuwe reviews toegevoegd worden. Hierdoor zal het bevragen nog sneller zijn. In de praktijk kunnen we een efficiënte Yelp-recommender maken.

\subsection{Evaluatie clustering}
\label{sub:chapt4_eval_clustering}
% TODO voordelen/nadelen van technieken
% geen expliciet gebruik van profielen bij clustering (en omgekeerd indien approx) => niet voldoende om enkel dit te nemen als evaluatie metriek
% TODO schalen de clusteringsmetrieken? -> nee, maar is het nodig om dit op zoveel data te doen => kleiner delen van de data (checken wat het verband is)
% => DUNN schaalt helemaal niet (maar kan aan implementatie liggen dus skip)

% metrics vergelijken

% todo potentielepotentielepotentielepotentielepotentielepotentiele verklaring -> embeddings zijn niet gefinetuned op eten => overlap CLUSTERINGSMETRICS


\subsection{Gebruikers- en restaurantprofielen}
\label{sec:chapt4_nlp_profielen}
De uiteindelijke features die we in het model gebruiken worden opgesteld op basis van het BERTopic model. Deze features bestaan uit een vector voor elke gebruiker en elk restaurant, we noemen ze respectievelijk gebruikers- en restaurantprofielen. Het BERTopic model zelf maakt gebruik van de volledige dataset, dit betekent niet dat de profielen dit ook doen. Deze worden met een deel van de data opgesteld, zoals beschreven in \autoref{sec:chapt4_data_flow}. \newline
Het opstellen kan op twee manier gebeuren. De eerste manier spreekt voor zich, deze maakt simpelweg gebruik van het model door de reviews zelf te clusteren en op basis daarvan een profiel op te stellen. De andere manier zal geen gebruik maken van de clustering. Deze zal een profiel afleiden uit de topic representaties van het getrainde BERTopic model. Ten slotte maken we een onderscheid tussen een gebruikers en een restaurant. Dit komt omdat de stappen om het profiel op te stellen licht kunnen wijzigen door bijvoorbeeld sentiment analysis toe te passen.


\subsubsection{Profiel op basis van de clustering}
\label{sub:chapt4_profile_by_clustering}
% VIA CLUSTERING
% clusters ci bepalen voor elke zin => CACHED
% tellen hoeveel keer elke cluster gekozen werd per user:
% TOEVOEGING sentiment: positief/negative => +1 en -1 of gewogen via de confidence score -> performance met/zonder (MOET via neuraal netwerk)
% groupby reviews => KEUZE normalizeren VS later doen -> elke review evenveel impact vs elke zin evenveel impact
% indien genormalizeerd => gemiddelde nemen van alle reviews van de user => elke review heeft evenveel impact
% indien niet genormalizeerd => alles sommeren en dan normalizeren => gevolg is dat elke zin evenveel impact heeft.
De eerste stap van deze methode is het verkrijgen van de clustering. We gaan voor elk document $d_i$ een cluster $c_j$, met $0 \le j < k$, toekennen aan de hand van het model met $k$ clusters zoals afgebeeld in \autoref{fig:chapt4_documents_to_clustering}. 

\mijnfiguur[H]{width=12cm}{fig/chapt4/NLP/documents_to_clustering.jpg}{Het bepalen van de clustering van de documenten door de bevraging van een getraind BERTopic model.}{fig:chapt4_documents_to_clustering}

Om vervolgens een profiel op te stellen willen we weten wat de meest voorkomende onderwerpen zijn. We moeten dus deze clusters aggregeren per gebruiker of restaurant. Hiervoor gaan we eerst de clusters $c_j$, verkregen per zin, samenvoegen tot een vector $R_r$ die de clustering van één review $r$ zal voorstellen. We doen dit door een vector van lengte $k$, met $k$ het aantal clusters van het gebruikte model, op te stellen. Elk element $x_j$, met $0 \le j < k$, komt overeen met het aantal voorkomens van $c_j$ bij de zinnen van de review. Deze vector wordt voorgesteld in \autoref{eq:chapt4_profile_per_review}

\begin{equation}
\label{eq:chapt4_profile_per_review}
    R_r = [x_{r0}, x_{r1}, ..., x_{rk}]
\end{equation}

Een variant hierop kan gegenereerd worden door gebruik te maken van sentiment analysis. Om dit te realiseren gaan we eerst voor elk document $d_i$ bepalen of deze positief of negatief is, hierbij zal ook een zekerheidsscore $s_i$ gegenereerd worden. Nu is $x_{rj}$ niet langer gelijk aan het aantal voorkomens van $c_j$, we zullen nu de som nemen van deze voorkomen met de overeenkomstige $s_i$. Indien het document als negatief geclassificeerd is gebruiken we $-s_i$. Op deze manier kan elke vector $R_r$ aangepast worden door elke $x_{rj}$ aan te passen zoals hierboven beschreven en voorgesteld in \autoref{eq:chapt4_profile_per_review_with_sentiment}. Hierdoor kan een bepaald onderwerp een negatieve score krijgen, waarom dit nuttig kan zijn wordt beschreven in \autoref{sub:chapt4_users_vs_restaurants}.

\begin{equation}
\label{eq:chapt4_profile_per_review_with_sentiment}
    x_{rj} = \sum_{s \in POSITIVE_{rj}}{s} + \sum_{s \in NEGATIVE_{rj}}{-s}
\end{equation}

Nu we een profiel voor elke review hebben opgesteld kunnen we deze aggregeren tot één profiel per gebruiker of restaurant. We doen dit door de profielen van alle reviews van één bepaalde gebruiker of restaurant elementsgewijs op te tellen. Voor gebruiker $u$ nemen we de som van alle profielen van de reviews geschreven door  $u$. Ten slotte zullen we dit profiel normaliseren zodat alle waarden in het interval $[0-1]$ liggen, dit proces wordt beschreven in \autoref{eq:chapt4_profile_per_user}. We kunnen dit analoog doen voor een restaurantsprofiel, in dit geval gaan we sommeren over de profielen van alle reviews die geschreven zijn over het bepaalde restaurant.
% TODO normaliseren -> tussen [0,1] EN min=0, max=1 (is dit standaard of toch vermelden voor de duidelijkheid)

\begin{equation}
\label{eq:chapt4_profile_per_user}
    UP_{u} = normalize(\sum_{review \in R_u}R_{review})
\end{equation}

% TODO @ARNO: help met argumenteren dat één review evenveel impact moet hebben in plaats van één zin 
Aan de hand van bovenstaande methode zal elke zin evenveel impact hebben op het uiteindelijke profiel. Helaas is dit niet optimaal, want in theorie zal één review over één bezoek aan een restaurant gaan. Hierdoor is het logisch om te zorgen dat elke review evenveel impact maakt op een profiel, maar omdat verschillende reviews een ander aantal zinnen hebben is dit niet het geval. Het probleem is dat de vectoren $R$ met reviews met meer zinnen harder doorwegen wegens het gebruik van aantallen. Dit probleem wordt opgelost door elke vector $R$ te normaliseren zoals beschreven in \autoref{eq:chapt4_profile_per_review_normalized}. Het gevolg hiervan is dat elke review een even grote impact heeft op het uiteindelijke profiel.

\begin{equation}
\label{eq:chapt4_profile_per_review_normalized}
    R_r = normalize([x_{r0}, x_{r1}, ..., x_{rk}])
\end{equation}

\subsubsection{Profiel op basis van de representaties}
% VIA APPROXIMATION
% uit de representaties van de cluster VS document (MERK OP DAT DIT OOK KAN PER REVIEW, wij hebben het per zin gedaan (sommige zinnen meerdere onderwerpen)  => CACHED
% estimaties voor elke topic = SOM is 1
% neem de beste N per topic (door de anderen op 0 te zetten)
% OPTIONELE STAP -> normalizatie zodat de som van deze topics 1 is
% analoog aggregeren per review en dan per user/restaurant
% ten slotte het elk profiel normalizeren (tussen 0-1)

% voordeel => NOG meer topics per zin!
Met deze methode maken we geen gebruik van de clustering, maar van de representatie van de topics van het model. Hierbij zal men elk document $d_i$ opsplitsen in meerdere groepen van woorden aan de hand van een sliding window. Vervolgens gaat men voor elke groep de c-TF-IDF representatie berekenen en bekijken hoe gelijkaardig deze zijn aan de mogelijk topic representaties van het model zelf. Deze vergelijking zal gebeuren via de cosinusgelijkenis. Hierna zal men alle groepen elementsgewijs sommeren en ten slotte normaliseren zodat de som van de elementen van de vector gelijk is aan één. De uiteindelijke output is een genormaliseerde vector $A$ met lengte $k$, waarbij $k$ het aantal clusters van het model is. Hierbij stelt de waarde $A_j$ de relevantie van topic $c_j$ voor, deze waarde is relatief tegenover de andere topics.

\mijnfiguur[H]{width=12cm}{fig/chapt4/NLP/documents_to_approximation.jpg}{Het bepalen van een approximatie voor de documenten door gebruik te maken van de representaties van een getraind BERTopic model.}{fig:documents_to_approximation}

Bij de verschillende BERTopic modellen heeft $k$ een verschillende waarde, hierdoor zullen de waarden in de vector $A$ kleiner/groter zijn wegens een grotere/kleinere spreiding. Aangezien we elk profiel met een gelijkaardige impact per review willen opstellen, zullen we elke vector $A$ aanpassen. We doen dit door de $n$ hoogste waarden bij te houden en de overigen gelijk te stellen aan nul. Vervolgens zullen we deze vector normaliseren zodat elke waarde in het interval $[0,1]$ ligt, hierdoor zal de impact niet afhangen van de lengte van de vector. Dit proces is beschreven in \autoref{eq:chapt4_approx_normalize_sentence_profile}. 

\begin{equation}
AN = normalize([an_0, an_1, ..., an_k]) \;\;\;\;\;\;\;
\label{eq:chapt4_approx_normalize_sentence_profile}
\begin{cases}
    an_i = a_i, & \text{als } a_i \in \text{top n}  \\
    an_i = 0,   & \text{anders}
\end{cases}
\end{equation}

Met deze aangepaste vectoren $AN$ kunnen we de reviewprofielen opstellen. We zullen dit doen door elementsgewijs de som te nemen van elke $AN_i$ overeenkomend met de zinnen van de review. Deze reviewprofielen gaan we ook normaliseren zodat elke review evenveel impact heeft, deze berekening wordt weergegeven in \autoref{eq:chapt4_profile_per_review_approx}.

\begin{equation}
\label{eq:chapt4_profile_per_review_approx}
    RA_{r} = normalize(\sum_{zin \in R_r}AN_{zin})
\end{equation}

Aan deze reviewprofielen kunnen we ook sentiment analysis toevoegen. Hierbij zullen we een gewogen som nemen tussen de som van elke positieve zin en met de som van elke negatieve zin. De gewichten van de positieve en negatieve vector zijn respectievelijk $1$ en $-1$. Deze berekening wordt weergegeven in \autoref{eq:chapt4_approx_with_sentiment}.

\begin{equation}
\label{eq:chapt4_approx_with_sentiment}
    RA_{r} = normalize(\sum_{AN \in POSITIVE_{r}}{AN} - \sum_{AN \in NEGATIVE_{r}}{AN})
\end{equation}

Ten slotte zullen we de gebruikers- en restaurant profielen opstellen vanaf de reviewprofielen. Dit zal analoog gebeuren als in \autoref{sub:chapt4_profile_by_clustering} aan de hand van \autoref{eq:chapt4_profile_per_user}.

\subsubsection{Gebruikers tegenover restaurants}
\label{sub:chapt4_users_vs_restaurants}
In de vorige secties haalden we verschillende mogelijkheden aan op gebruikers- en restaurantprofielen op te stellen. Voor de meeste parameters is er weinig tot geen verschil tussen een profiel opstellen voor een gebruiker tegenover een restaurant.
Dit geldt ook voor de algoritmes op basis van clustering en representaties, deze delen worden besproken in \autoref{sub:chapt4_nlp_resultaten}.

Het meest significante verschil zit bij de sentiment analysis. Beschouw het volgende voorbeeld, een gebruiker gaat naar een pizzarestaurant. De gebruikerservaring was negatief aangezien de pizza niet lekker was, hierdoor is de sentiment analysis van de review negatief. Indien we dit toevoegen aan een gebruikersprofiel zou dit weergeven dat een gebruiker pizza niet lekker vindt. Deze assumptie komt niet overeen met de realiteit, aangezien de gebruiker naar een pizzarestaurant nemen we aan dat hij pizza lekker vindt. Hierdoor is het logisch om geen sentiment analysis toe te passen bij een gebruikersprofiel. \newline
Beschouw hetzelfde voorbeeld voor een pizzarestaurant die, wat betreft het eten, meerdere negatieve reviews heeft. In dit geval zouden negatieve scores in het restaurantprofiel voorstellen dat de pizza niet lekker is. Dit komt dus wel overeenkomen met de realiteit dat gebruikers de pizza niet smaakvol, zoals beschreven in de reviews. Hierdoor zullen we bij het opstellen van een restaurantprofiel wel gebruik maken van sentiment analysis.

Een ander verschil is de relevantie van de verschillende onderwerpen. Voor gebruikers en restaurants zullen deze %todo overlap + verschil

\subsection{Resultaten}
\label{sub:chapt4_nlp_resultaten}
% todo vergelijk resultaten hier
%   => REFEREER HOE WE VALIDEREN -> VIA NEURAAL NETWERK
% offline vs ONLINE => duidelijk online beter: daarom vanaf hier enkel online modellen

% sentiment VS geen sentiment in topics + gebruiker vs restaurant
% sentiment VS geen sentiment in approx + gebruiker vs restaurant (zien we zelfde trend?)

% aantal topics = lengte model = ? impact

% binnenin approx: top_n = ? (HANGT DIT AF VAN LENGTE MODEL!)

% guided topics => ? geen verbetering (zoals vermeld)
% brol_topics filteren => ? geen verbetering + conclusie: broltopics in principe niet erg => neuraal netwerk leert dat er uit filteren => komt uit conclusie manueel filteren topics


\subsubsection{Verband tussen verschillende evaluatiemethoden}
\label{sub:chapt4_compare_eval_methods}
% todo verband tussen clusteringmetrics en RMSE

\section{Neuraal netwerk}
\label{sec:chapt4_neuraal_netwerk}

% TODO: waarom een neuraal netwerk? Welke voordelen spreken ons aan?
% TODO: Arno, go ahead. Bespreek in deze section ook al de individuele resultaten. Random grafieken: go! Chapter 5 is enkel bedoelt om alles nog eens samen te vatten
\subsection{Input}
Zoals aangeduid in \autoref{fig:chapt4_architectuur_begin} zijn er twee bronnen van inputdata voor het neurale netwerk: de labels rechtstreeks geëxtraheerd uit de Yelp dataset, en de geschreven reviews die zijn omgezet naar numerieke features zoals beschreven in \autoref{sec:chapt4_nlp_profielen}. Beide bronnen modelleren steeds zowel een gebruikerprofiel als een restaurantprofiel. Met deze data moet het neuraal netwerk een voorspelling maken welke score die specifieke gebruiker aan dat specifieke restaurant geeft. Merk op dat deze profielen worden opgesteld met slechts een deel van de train- of testset, zoals beschreven in \autoref{sec:chapt4_data_flow} % TODO: een zelfde soort 'warning' bij Arnoud zijn deel zetten

Een neuraal netwerk aanvaardt enkel numerieke features. Bij de NLP gebruikers- en restaurantprofielen is dit reeds opgelost. Bij de labels is er meer werk. We bespreken eerst hoe een restaurant gemodelleerd wordt, en daarna hoe we deze modellering kunnen aanpassen om ook gebruikersdata te ondersteunen.
% TODO: checken dat er niet te veel overlap is met Arnoud hier
\subsubsection{Restaurantlabels}
\label{sec:chapt4_nn_restaurantlabels}
Een restaurant wordt hoofdzakelijk beschreven in de Yelp dataset met behulp van categorieën en attributen (\autoref{sec:chapt3}). We maken gebruik van one-hot encoding om de aanwezigheid van een categorie (nominaal) bij een restaurant aan te duiden. Doordat er in de totale dataset 1311 unieke categorieën zijn, zou er door de one-hot encoding een zeer ijle inputvector gemaakt worden. Om dit probleem te beperken, houden we enkel de categorieën die bij minstens 500 restaurants (1\% van totaal) voorkomen. Zo houden we nog 75 categorieën over.

Niet ieder restaurant beschikt over een waarde voor ieder attribuut. Ongeveer 33\% ontbreekt minstens één attribuut. Ordinale data, zoals de prijsklasse, wordt omgezet naar een waarde in $[0, 1]$ die overeenkomt met de rangorde. Voor de andere attributen gebruiken we opnieuw one-hot encoding. In tegenstelling tot categorieën moeten we rekening houden dat de afwezigheid van een attribuut in de dataset niet per se overeenkomt met het werkelijk ontbreken van dat attribuut in de echte wereld. Aangezien deze data wel waardevol lijkt voor aanbevelingen, lossen we dit probleem op door een standaardwaarde per attribuut in te stellen. Zo wordt de afwezigheid van een one-hot encoded attribuut gemodelleerd als $0,5$.

We berekenen de huidige gemiddelde score van een restaurant en voegen deze toe als feature. De Yelp dataset bevat ook de check-ins voor ieder restaurant. Deze datapunten worden omgezet tot een numerieke waarde door het gemiddeld aantal check-ins per week te berekenen. Samen modelleren deze twee nieuwe features een rudimentaire vorm van de populariteit van een restaurant.

Het gemiddeld aantal check-ins per week bij een restaurant varieert sterk, van quasi 0 bij niche restaurants tot meer dan 100 bij grote ketens. Deze afgeleide feature ligt dus niet in $[0, 1]$ zoals alle andere features en zou ervoor zorgen dat deze feature meer doorweegt in het neurale netwerk. De spreiding van waarden van deze feature is ook niet uniform: er zijn enkele uitschieters die niet in lijn liggen met de overige restaurants. Een gewone normalisatie zal dit dus niet oplossen. We herschalen daarom alle data die tussen het 5$^e$ en 95$^e$ percentiel tot $[0, 1]$ en extreme lage en hoge waarden transformeren we tot respectievelijk $0$ en $1$. Hierdoor krijgen we een betere spreiding en heeft deze feature een even groot gewicht als alle andere.

\subsubsection{Gebruikerslabels}
Uit \autoref{sec:chapt4_nn_restaurantlabels} volgt dat we een restaurant $i$ kunnen voorstellen aan de hand van labels in de vorm van een vector $RP_i = (feature_1, feature_2, ..., feature_n)$. We bepalen nu per gebruiker de verzameling van restaurants $\mathcal{V}$ waarvoor die gebruiker een review heeft achtergelaten. We kunnen dan voor gebruiker $j$ een gebruikersprofiel $UP_j$ opstellen:
\begin{equation}
    UP_j = \frac{\sum_{v \in \mathcal{V}} (RP_v \cdot R_{j, v})}{\vert \mathcal{V} \vert}
\end{equation}

met $R_{j, v}$ de genormaliseerde score die gebruiker $j$ geeft aan restaurant $v$. De Yelp dataset bevat nog enkele andere gegevens over de gebruikers: zo wordt er bijgehouden hoeveel keer andere gebruikers een review 'nuttig', 'grappig' of 'cool' vonden. Deze metadata over een gebruiker helpt om de betrouwbaarheid van de reviews te modelleren. We creëren een nieuwe feature door de som van het aantal positieve interacties te nemen. Er zijn enkele bekende gebruikers op Yelp, die veel volgers hebben en hierdoor veel meer feedback hebben gekregen op hun reviews. Daarom herschalen we eerst de feature door alle waarden hoger dan het 99$^e$ percentiel te mappen op $1$ en de rest te normaliseren tussen $[0, 1]$.

We experimenteerden ook om de gemiddelde score van een gebruiker uit de trainset toe te voegen aan de inputvector. Echter merkten we op dat deze feature zeer dominant is tijdens training, en het model zich hier volledig naar aanpast. Om dit te voorkomen, kozen we ervoor om deze feature niet in de inputvector te steken.

Beide profielen, gecombineerd met de profielen die gecreëerd zijn aan de hand van NLP, vormen de input voor het neurale netwerk. (\autoref{fig:chapt4_architectuur_input})

\mijnfiguur[H]{width=8cm}{fig/chapt4/predictor/architectuur_input.png}{Close-up inputlaag van architectuur zoals in \autoref{fig:chapt4_architectuur_begin}}{fig:chapt4_architectuur_input}


% TODO: zie correlatiegraphs, om te bekijken of er obvious dubbele data inzit of niet
% TODO: zeggen dat corr. voor andere NLP profielen er basically zelfde uitziet

\subsubsection{Correlaties}
Het lijkt overbodig om zowel de labels als geschreven reviews te verwerken in hetzelfde neurale netwerk, daar deze dezelfde objecten omschrijven. Intuïtief zou men denken dat er sterke correlaties bestaan tussen een restaurantprofiel opgesteld door NLP en aan de hand van labels. Dit blijkt echter niet het geval: er zijn maar vijf  van de TODO paren van features waarbij de absolute waarde van de correlatie groter is dan $0,75$. Nog opmerkelijker: deze features zijn steeds gebaseerd op de gelabelde data.
% TODO: de TODO in de tekst invullen!
% TODO: afbeelding van correlatiegrafieken toevoegen


\begin{table}[H]
    \centering
    \begin{tabular}{l|l}
    Feature 1 & Feature 2 \\ \hline
    user\_compliments & user\_fans \\
    user\_positive\_interactions & user\_compliments \\
    user\_positive\_interactions & user\_fans \\
    user\_category\_nightlife & user\_category\_bars \\
    user\_category\_beer & user\_category\_wine\_\&\_spirits
    \end{tabular}
    \caption{Paren van features waarbij de absolute waarde van de correlatie groter is dan $0,75$}
    \label{}
\end{table}

De volledige data is beschikbaar in \verb|src/corr.html|. Deze conclusie geldt voor alle varianten van profielen gemaakt door NLP.

\subsection{Testset-up}
% TODO: hoe evalueer je wat een goed model is? Hoe doen we dit objectief? Welke problemen had je met het opstellen van een objectief framework?
De modellen zijn steeds geschreven in Python 3.10. De implementaties maken gebruik van PyTorch 2.0.0. \cite{pytorch} De inputvector voor het neurale netwerk bestaat uit \verb||torch.float32 getallen. De uitvoering gebeurt op een AMD Ryzen 5800X, NVIDIA RTX 2080, en 64GB werkgeheugen. % TODO: checken dat bij arnoud dit klopt, en eventueel checken dat arnoud ook de azure pc erop zet, want dat lijkt beter

We analyseren iedere implementatie op dezelfde manier. De data wordt verwerkt zoals beschreven in \autoref{sec:chapt4_data_flow}: iedere epoch trainen we het neuraal netwerk met een trainset en berekenen we vervolgens de loss op de testset. Als de loss meerdere epochs op rij omhoog gaat, stoppen we het trainen om overfitting te voorkomen. De gebruikte lossfunctie voor optimalisatie is de MSE.

Nadat het model volledig getraind en getest is volgens de hierboven beschreven methode, analyseren we het model aan de hand van voorspellingen zoals we ze in de praktijk zouden gebruiken: we nemen een subset van de testset om aan de hand van het model een score tussen 1 en 5 te voorspellen. We ronden hierbij de voorspelling af, zodat het domein van de voorspelde score exact overeenkomt met het domein dat een gebruiker heeft op Yelp. We maken een histogram van het verschil tussen de voorspelling en echte score toegekend door de gebruiker. Dit histogram gebruiken we als controle dat een lagere MSE overeenkomt met meer accurate voorspellingen.

\subsection{Basisimplementatie}
\label{sec:chapt4_basisimplementatie}
We beginnen met een basisimplementatie van een neuraal netwerk, om te verifiëren dat het mogelijk is om de echte score te schatten op basis van de inputdata. We werken met een standaard multilayer perceptron (feed-forward) model. De output van het model is één waarde in $[0, 1]$. We implementeren het aanbevelingsysteem dus als een regressiemodel, waarbij de output daarna herschaald wordt naar een score tussen 1 en 5. We kiezen voor de bij regressiemodellen veelgebruikte lossfunctie MSE. Deze is ook makkelijk om te zetten naar RMSE, wat vergelijken met andere andere onderzoeken rechtdoorzee maakt. \cite{narre, deepconn, wide_deep_learning_paper}
% TODO: eerste idee en resultaten, problemen en opmerkingen...
Het model bestaat uit 4 verborgen lagen, waarbij het aantal neurons per laag steeds halveert. In tegenstelling tot collaborative filtering, kan dit model wel niet-lineaire verbanden capteren. We gebruiken Stochastic Gradient Descent (SGD) als netwerkoptimizer, met een learning rate van $0,01$.(\autoref{fig:chapt4_basisimplementatie})

\mijnfiguur[H]{width=16cm}{fig/chapt4/predictor/basisimplementatie_netwerk.png}{Voorstelling basisimplementatie, waarbij $n$ het aantal inputfeatures voorstelt}{fig:chapt4_basisimplementatie}



\subsection{Uitbreidingen architectuur}
We blijven bij een standaard multilayer perceptronmodel. Dit is de meest logische keuze voor deze context. Er is bijvoorbeeld geen nood aan een model met geheugen zoals LSTM daar de inputdata geen volgorde bevat. 
We onderzoeken wel het aantal lagen en het aantal neuronen per laag. Hoe complexer het netwerk, hoe meer verbanden het capteren. Echter zorgt een stijging in complexiteit van het netwerk ook voor een complexere training die meer data nodig heeft. Complexere netwerken scoren ook slechter bij explainability. Daarom voeren we experimenten uit met een eenvoudiger netwerk met maar 1 verborgen laag, tot 8 verborgen lagen. Deze netwerken volgen een analoge structuur zoals beschreven in \autoref{sec:chapt4_basisimplementatie}. Vanaf 5 verborgen lagen zal de eerste verborgen laag wel $20$\% groter zijn dan de inputlaag, om het netwerk de kans te geven om complexere verbanden te modelleren. De tussen de eerste 3 verborgen lagen zit steeds een dropout-laag. Deze laag heeft een 20\% kans om een neuron op 0 te zetten en helpt zo om overfitting te voorkomen. \cite{nn_dropout} Tijdens het testen van het model worden deze dropout-lagen uitgezet.

\subsection{Optimalisaties}
\subsubsection{Netwerk-optimizer}
% TODO: hoeveel hiervan moet in hs2 staan?
SGD is een zeer eenvoudige 
% TODO: optimizer, custom loss function
% TODO: learning rate optimalisatie

% TODO: maak een note dat de input het belangrijkste is voor de performance van het netwerk, en niet de architectuur zelf enzo
% TODO: uitleg grid search



\subsection{Random Forest}
% TODO: random forest werkt ok, als we explainability belangrijk zouden vinden. RMSE van 1.2

\subsection{Resultaten}
\subsubsection{Architectuur neuraal netwerk}
% TODO: simpel is altijd default prediction, geef graph van altijd 4 voorspellen + rmse!
% TODO: moeilijker is random, geef graph van random + rmse
% TODO: geef graph van het uiteindelijke beste model + rmse, maar hoe zit het dan met LR en optimizer optimalisatie etc?
% TODO: MLP, simpel, moeilijker, custom loss functie. Bij makkelijke architectuur gaan we naar default predicition van 4 sterren, onafhankelijk van optimizer en loss functie (penalty/weights). Bij moeilijkere architectuur hebben we moeite om loss naar beneden te krijgen.

% TODO: sim
\subsubsection{Cold-startprobleem}
% TODO: Testset aanpassen om enkel users met veel reviews te beschouwen

\subsubsection{Grootte dataset}
% TODO: NADAT we opsplitsen bij users, een deel van de users weggooien bij de train kant (en bijhorende reviews opnieuw uitrekenen via join), zal ook sneller trainen dan

\chapter{Resultaten}
% TODO: Hier het eindresultaat nog eens neerploppen, en eventueel de online-analyse-resultaten van onze enquete? Bespreek waarom het goed of slecht is, wat de sterke en zwakke kanten zijn.
\section{Vergelijking met andere methoden}
% TODO: DeepCoNN, traditionele methoden, oof dat gaat veel werk zijn.


% TODO: apart hoofdstuk voor discussie? Te bespreken

\chapter{Conclusie}
% TODO: conclusie
\section{Toekomstig werk}
% TODO: toekomstig werk
% TODO: meer specifiek SBERT model getraind op restaurant data/culinaire data zou performance van topics beter moeten maken, maar deze dataset ontbreekt voor ons

% ------------ REFERENCES ------------
% Here you have your bibliography created


\addcontentsline{toc}{chapter}{Bibliografie} %show bibliografie in TOC
\printbibliography[notkeyword={eng_summary}]
% Here you insert your appendices
\appendix

\begin{appendices}

% Dit voegt het woord Bijlage toe aan de titel!
\titleformat{\chapter} % command
  [display] % shape
  {\fontsize{18}{22} \selectfont \coltitle } % format
  {\MakeUppercase{\chaptertitlename \ \thechapter}} % the label
  {-2ex} %separator space
  {\fontsize{24}{32} \selectfont \bf \raggedright \MakeUppercase{\uline{#1}}} %before code
  { } %aft%after code


\include{app1}

\end{appendices}

\end{document}

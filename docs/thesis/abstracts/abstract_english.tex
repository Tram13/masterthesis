{\titlespacing*{\chapter}
{0pt}{-5pt}{-5pt}\chapter[Summary]{}}
\rule{\linewidth}{0.5mm}
\hspace*{0pt}
\section*{Design of a machine learning algorithm for restaurant \\recommendations based on labeled and textual data}
\rule{\linewidth}{0.5mm}
% \chapter[Summary]{Design of a machine learning algorithm for restaurant \\recommendations based on labeled and textual data}
% TODO: Zie 'thesis_van_bert.pdf' als voorbeeld
% TODO: volgens UGent styleguide is dit 2 tot 6 pagina's. Ik zou toch zeker voor 5 pagina's gaan, indien mogelijk, want we zijn met 2 personen.
% TODO: controleer dat alles in het Engels staat, inclusief referenties, 'deel-deelparagraaf', 'Figuur', ...
% TODO: de nummering van hoofdstukken is manueel in dit deel!!!!! controleer dat deze correct lijkt, en ook in de inhoudsopgave er niet in staat!
% TODO: enkel de meest belangrijke grafieken van resultaten mogen hier in, zeer beknopt!
% TODO: BRONVERMELDING:
% TODO:     --> HERGEBRUIK GEEN BRONNEN UIT DE MAIN THESIS PAPER
% TODO:     --> INSTEAD: DEFINIEER DE NODIGE BRONNEN OPNIEUW IN Thesis_bib.bib, met een andere ID. VOEG PER BRON DEZE LIJN TOE: keywords = {eng_summary}
% TODO:         --> ZO WORDT: @book{cursus_hs2, title={TAXONOMY AND ALGORITHM OVERVIEW}, publisher={Universiteit Gent}, author={De Pessemier, Toon and Martens, Luc}}
% TODO:         --> @book{cursus_hs2_eng_summary, keywords = {eng_summary}, title={TAXONOMY AND ALGORITHM OVERVIEW}, publisher={Universiteit Gent}, author={De Pessemier, Toon and Martens, Luc}}
% TODO:                              ‾‾‾‾‾‾‾‾‾‾‾  ‾‾‾‾‾‾‾‾‾‾‾‾‾‾‾‾‾‾‾‾‾‾‾‾

\textbf{Students:}
\hfill
\textbf{Supervisors:} \\
Arnoud De Jonge
\hfill
prof. dr. ir. Toon De Pessemier \\
Arno Vermote
\hfill
prof. dr. ir. Luc Martens \\
\\
\hspace*{0pt}\hfill Master Computer Science


\begin{multicols}{2}
[
\section*{\centering Abstract}
% TODO: abstract
Hier staat een abstract dat bestaat uit meedere lijnen want ik moet hier iets verzinnen dat ongeveer een beetje lijkt om een tekst om een realistisch beeld te geven of het deftig gecentreerd staat of niet.
]

\section*{1. Introduction} % Introduction, maar de titel mag er niet geschreven staan!
% TODO: Inleiding van probleem
% TODO: Probleem- en doelstelling
TODO: Hier staan een inleiding met een voorbeeldcite: \cite{deepconn_eng_summary}. Deze inleiding is ook een beetje random gezever om het te laten lijken op een mooi ingevulde tekstblok zodat de formatting mooi gecontroleerd kan worden door mij.

\section*{2. Methodologies}
% TODO: NLP, en hoe we uiteindelijk de tekst omzetten naar features, zeg hier *iets* maar weinig over over """gefaalde""" experimenten, zoals offline, approximation, ..
% TODO: MLP: welke architectuur?
n tegenstelling tot wat algemeen aangenomen wordt is Lorem Ipsum niet zomaar willekeurige tekst. het heeft zijn wortels in een stuk klassieke latijnse literatuur uit 45 v.Chr. en is dus meer dan 2000 jaar oud. Richard McClintock, een professor latijn aan de Hampden-Sydney College in Virginia, heeft één van de meer obscure latijnse woorden, consectetur, uit een Lorem Ipsum passage opgezocht, en heeft tijdens het zoeken naar het woord in de klassieke literatuur de onverdachte bron ontdekt. Lorem Ipsum komt uit de secties 1.10.32 en 1.10.33 van "de Finibus Bonorum et Malorum" (De uitersten van goed en kwaad) door Cicero, geschreven in 45 v.Chr. Dit boek is een verhandeling over de theorie der ethiek, erg populair tijdens de renaissance. De eerste regel van Lorem Ipsum, "Lorem ipsum dolor sit amet..", komt uit een zin in sectie 1.10.32.

\section*{3. Experimental Setup}
% TODO: hoe we objectief vergelijken met andere implementaties van gebruikers- en restaurantprofielen NLP (testmethode)
% TODO: hoe netwerkarchitecturen en parameters vergelijken?
n tegenstelling tot wat algemeen aangenomen wordt is Lorem Ipsum niet zomaar willekeurige tekst. het heeft zijn wortels in een stuk klassieke latijnse literatuur uit 45 v.Chr. en is dus meer dan 2000 jaar oud. Richard McClintock, een professor latijn aan de Hampden-Sydney College in Virginia, heeft één van de meer obscure latijnse woorden, consectetur, uit een Lorem Ipsum passage opgezocht, en heeft tijdens het zoeken naar het woord in de klassieke literatuur de onverdachte bron ontdekt. Lorem Ipsum komt uit de secties 1.10.32 en 1.10.33 van "de Finibus Bonorum et Malorum" (De uitersten van goed en kwaad) door Cicero, geschreven in 45 v.Chr. Dit boek is een verhandeling over de theorie der ethiek, erg populair tijdens de renaissance. De eerste regel van Lorem Ipsum, "Lorem ipsum dolor sit amet..", komt uit een zin in sectie 1.10.32.

\section*{4. Results and Discussion}
% TODO: Wat zijn de uiteindelijke resultaten? + potentieel een graph hier? idk???? Wat doen andere thesissen die hun resultaten hier vermelden?
% TODO: vergelijking met bestaande methoden
% TODO: conclusie (eventueel in aparte section)
n tegenstelling tot wat algemeen aangenomen wordt is Lorem Ipsum niet zomaar willekeurige tekst. het heeft zijn wortels in een stuk klassieke latijnse literatuur uit 45 v.Chr. en is dus meer dan 2000 jaar oud. Richard McClintock, een professor latijn aan de Hampden-Sydney College in Virginia, heeft één van de meer obscure latijnse woorden, consectetur, uit een Lorem Ipsum passage opgezocht, en heeft tijdens het zoeken naar het woord in de klassieke literatuur de onverdachte bron ontdekt. Lorem Ipsum komt uit de secties 1.10.32 en 1.10.33 van "de Finibus Bonorum et Malorum" (De uitersten van goed en kwaad) door Cicero, geschreven in 45 v.Chr. Dit boek is een verhandeling over de theorie der ethiek, erg populair tijdens de renaissance. De eerste regel van Lorem Ipsum, "Lorem ipsum dolor sit amet..", komt uit een zin in sectie 1.10.32.











\printbibliography[heading=subbibliography, keyword={eng_summary}]
\end{multicols}

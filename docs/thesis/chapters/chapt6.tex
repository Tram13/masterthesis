\chapter{Conclusie}

De combinatie van tekstuele en gelabelde data stelt ons in staat om beter reviewscores voor restaurants te voorspellen. Het voorgegestelde algoritme is in staat om een lagere RMSE te scoren dan de state of the art-technieken die slechts één van de twee databronnen gebruiken. Meer specifiek gebruiken we BERTopic om de zinnen van geschreven reviews op te delen in clusters. Deze clusters stellen verschillende onderwerpen voor, zoals \q{pizza} of \q{fastfood}. We verzamelen dan alle onderwerpen per gebruiker en per restaurant, om zo een beschrijvende vector (profiel) op te stellen. Door deze clustering op verschillende manieren op te stellen en aan te bieden aan een neuraal netwerk, vonden we dat een online BERTopic-algoritme met 400 topics het beste presteerde. Het online model van 50 topics presteerde marginaal slechter, maar we kozen om met slechts 1 model verder te gaan. Een voordeel van een profiel met meer dimensies is dat een geavanceerder neuraal netwerk meer potentieel heeft om verborgen verbanden te vinden. Bij het opstellen van deze profielen kunnen we onafhankelijk van een BERTopic-model externe eigenschappen toevoegen. Wij hebben sentiment analysis toegevoegd aan de gebruikers- en restaurantprofielen. Hierdoor is het mogelijk om ook slechte ervaringen te modelleren. We namen waar dat de loss van het neuraal netwerk nog lager was als we enkel sentiment analysis toepassen bij het opstellen van de restaurantprofielen.

De beste implementatie van het neuraal netwerk haalt een MSE loss van 0.0771 op de genormaliseerde score, of een RMSE loss van 1.1107. Dit wil zeggen dat op een gemiddelde voorspelling, ons aanbevelingssysteem een fout maakt van net iets meer dan 1 ster. Zo zal een 5-ster review gemiddeld voorspeld worden als een 4-ster. De voorspellingen zijn minder accuraat voor nieuwe gebruikers (Cold-Startprobleem), maar de accuraatheid stijgt snel naarmate de gebruiker meer reviews achterlaat. We meten dat het neuraal netwerk efficiënt getraind en getest kan worden met slechts de helft van de beschikbare data, of dus ongeveer 2.3 miljoen reviews. De grootste zwakte van het voorgestelde model is het voorspellen van ondergerepresenteerde klassen: reviews met 1, 2 of 3 sterren.

Onze implementatie heeft een aanvaardbare uitvoeringstijd en geheugencomplexiteit om in een reële productieomgeving toegepast te worden. Het BERTopic-model moet maar eenmalig opgesteld worden, en kan online geüpdatet worden met nieuwe reviews. Ook het neuraal netwerk kan incrementeel extra getraind worden. Doordat we profielen op voorhand kunnen berekenen, zijn we in staat om in realtime voorspellingen te maken.

\section{Toekomstig werk}
\subsection*{Clusteringsperformantie \& Finetuning}
% TODO: zero shot classification
% TODO: meer specifiek SBERT model getraind op restaurant data/culinaire data zou performance van topics beter moeten maken, maar deze dataset ontbreekt voor ons
De resultaten van de clusteringsmetrieken zijn niet optimaal. Verder onderzoek naar betere clustering lijkt veel potentieel te hebben. Het is mogelijk om dieper in te gaan op de hyperparameters van de verschillende dimensionaliteitsreductie- en clusteringsalgoritmen, welke een onderdeel zijn van BERTopic. De meest logische aanpak is om een betere embedding te generen. Dit is namelijk de basis van het clusteren. Hiervoor kan er onderzocht worden of een embeddingsmodel, gefinetuned op restaurantsdata, een betere representatie van de zinnen kan genereren. Hierdoor bestaat de mogelijkheid dat de limitaties van een te generiek embeddingsmodel overkomen worden, met als gevolg dat de clusters minder overlappen.

\subsection*{Extra analyses op tekstuele data}
Zoals eerder vermeld hebben we sentiment analysis verwerkt in de profielen. Deze toevoeging staat los van het BERTopic-algoritme. Verder onderzoek naar andere analyses zoals bijvoorbeeld emotiedetectie, entity extraction, zero shot encoding, etc kan de modelleringskracht positief beïnvloeden: het is mogelijk dat een breed spectrum aan emoties meer nuance kan geven over bepaalde reviews. In het geval van entity extraction is het mogelijk om bepaalde restaurantgerelateerde termen zoals gerechten of keuken te detecteren. Ten slotte vermelden we zero shot encoding: hiermee kunnen we elke review classificeren en bepalen wat de gebruiker het belangrijkste vindt. Een mogelijke implementatie zou dan gewichten toevoegen aan de profielen indien een bepaalde gebruiker vooral geïnteresseerd is de kwaliteit van de gerechten en minder in service of vice versa.\newline
De resultaten van deze analyses kunnen in de gebruikers- en restaurantprofielen verwerkt worden of rechtstreeks als inputfeatures van het neuraal netwerk toegevoegd worden.

\subsection*{Ondergerepresenteerde klassen}
De grootste zwakte van het voorgestelde neuraal netwerk zijn de voorspellingen van reviews met 1, 2 of 3 sterren. Er zijn verschillende technieken voor dit probleem die wij nog niet onderzocht hebben. Een ensemblemodel dat bestaat uit ons standaard neuraal netwerk en een ander neuraal netwerk specifiek getraind op de klassen met een lage accuraatheid zou een significante verbetering kunnen opleveren.

\subsection*{Aanbevelingen maken voor een eindgebruiker}
Om het neuraal netwerk te gebruiken in een productieomgeving moeten de voorspellingen nog omgezet worden naar een lijst van aanbevelingen. Hiervoor berekent het aanbevelingssysteem de verwachte score voor ieder niet-bezocht restaurant en sorteert deze scores van hoog naar laag. De top-n best scorende restaurants worden dan aanbevolen. Het is mogelijk om de diversiteit te vergroten bij de aanbevelingen, door te zorgen dat de afstand van de restaurantprofielen van de aanbevolen restaurants groot is. Er moet dus een afweging gemaakt worden. Het onderzoek naar deze afweging valt niet meer binnen ons onderzoek.



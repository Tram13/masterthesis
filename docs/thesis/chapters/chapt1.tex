\chapter{Introductie}

"Aanbevelingssystemen zijn softwaretools en -technieken die aanbevelingen voor items voorzien die nuttig zijn voor de gebruiker. Deze aanbevelingen voorzien door het aanbevelingssysteem hebben de bedoeling de gebruikers te ondersteunen in het maken van keuzes, zoals welk item te kopen, welke muziek te beluisteren en welke nieuwsberichten te lezen." \cite{recsys_handbook}
Aanbevelingssystemen zijn alomtegenwoordig in ons dagelijks leven. Entertainmentproviders zoals Spotify en Netflix gebruiken al jaren aanbevelingssystemen om ons kennis te laten maken met nieuwe muziek en films. Google gebruikt het onder meer in Maps, om lokale bedrijven en horeca aan te rijken aan de gebruiker.

De keuze over welke producten relevant zijn voor een specifieke gebruiker gebeurt aan de hand van gegevens over de producten, en eventueel gegevens over de gebruiker. Stel het voorbeeld van een bioscoop: de producten zijn dan films. De bioscoop beschikt per product over gegevens die het genre van de film en het doelpubliek beschrijven. Voor de gebruiker houdt die een kijkgeschiedenis bij. Zo kunnen we voorkeuren leren uit de kijkgeschiedenis en deze als filter toepassen op het aanbod van films.

Er zijn verschillende motieven om aanbevelingssystemen te gebruiken. Als social media-platformen betere content aanbieden aan gebruikers zorgt dit voor een hogere screentime. Langer op TikTok scrollen betekent dat de gebruiker meer advertenties bekijkt, en dus meer inkomsten genereert. \cite{tiktokalgorithm}
Ook laat het de gebruiker kennis maken met long-tail items. Dit zijn de niche items die dus minder populair zijn, maar daarom niet minder kwalitatief. Doordat ze minder populair zijn, zijn ze wegens plaatsgebrek minder aanwezig in een fysieke winkel. Een online winkel heeft deze beperking niet. Er is echter zodanig veel keuze, dat de gebruiker overweldigd wordt. Deze enorme hoeveelheid aan producten creëert net een barrière voor de gebruiker. \cite{paradox_choice} Daarom is het slim filteren van items veel belangrijker geworden. Hierdoor zijn aanbevelingssystemen noodzakelijk bij grote webwinkels \cite{rise_of_recsys_in_ecommerce}. Zo kan steeds het optimale product weergegeven worden aan iedere individuele gebruiker. \cite{cursus_hs2}

Webwinkels kunnen aanbevelingssystemen ook gebruiken om commercieel interessantere items een voorkeur te laten genieten. Zo kunnen items met hogere winstmarges vaker aanbevolen worden.

\mijnfiguur[H]{width=12cm}{fig/chapt1/long_tail.png}{De long-tail}{fig:chapt1_long_tail}

\section{Probleemstelling}
Als een gebruiker op restaurant wilt gaan eten, beperkt die zich vaak tot de keuzes die hij al kent. Dit fenomeen kunnen we linken aan het verankeringseffect \cite{anchoring_effect}, waarbij een persoon te veel waarde hecht aan de enkele restaurants die hij al heeft uitgeprobeerd. Er zijn echter veel restaurants die zeer goed bij de gebruiker zouden passen, maar waar hij geen weet van heeft.

Diensten zoals Tripadvisor \cite{tripadvisor_algorithm}, Yelp of Google Maps helpen een gebruiker om deze keuze te maken. Deze diensten maken gebruik van aanbevelingssystemen om restaurants aan te bieden aan gebruikers op basis van diens locatie, voorkeuren en zoekterm. Echter zijn de systemen die gebruikt worden door de grote spelers niet feilloos \cite{recsys_bad, recsys_youtube_bad}. 

Er is nog weinig onderzoek naar aanbevelingssystemen die een combinatie van labels en vrije, tekstuele data zoals geschreven reviews gebruiken. Meer specifiek, er ontbreekt onderzoek naar aanbevelingssystemen die transformermodellen gebruiken om extra features toe te voegen aan een machine-learning model. Afzonderlijk onderzoek naar technieken die transformermodellen gebruiken om features te maken voor aanbevelingen bestaat al, maar die features worden niet gebruikt voor machine-learning systemen. \cite{masterthesis_nlp_italie} Aan de andere kant bestaan aanbevelingssystemen die volledig op machine learning gebaseerd zijn, maar geen transformermodellen gebruiken. \cite{deepconn} In deze thesis onderzoeken we of deze combinatie leidt tot een betere modellering van de voorkeuren van de gebruiker en of hieruit betere voorspellingen volgen. De onderliggende algoritmen van de belangrijkste en meest gebruikte technieken bespreken we in \autoref{sec:chapt2_huidige_technieken_aanbevelingssystemen}. 


\section{Doelstelling}
Ons onderzoek tracht met dezelfde beschikbare data als andere algoritmen een betere ervaring voor de gebruiker te creëren door beter in te kunnen schatten wat de gebruiker belangrijk vindt bij een restaurantbezoek. Deze betere ervaring komt neer op preciezer de score te voorspellen die de gebruiker aan een restaurant zou geven. Hiervoor gebruiken we een hybride aanbevelingssysteem, waarbij de geschreven reviews als extra databron worden gebruikt om meer informatie over de gebruiker te capteren. Met transformermodellen wordt deze extra data omgezet tot nieuwe numerieke features, die dan samengevoegd worden bij de gelabelde data (feature augmentation). Zo creëren we voorstellingen voor gebruikers en restaurants. Deze worden dan als input gebruikt voor een neuraal netwerk, waarbij de output een verwachte score voorstelt die de gebruiker aan dat restaurant zou geven.


Als we accuraat scores kunnen voorspellen, kunnen we de gebruiker de restaurants met de hoogste verwachte scores aanrijken, waardoor de gebruikerservaring en vertrouwen in het aanbevelingssysteem groeit.
